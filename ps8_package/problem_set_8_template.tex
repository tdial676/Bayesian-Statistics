% Options for packages loaded elsewhere
\PassOptionsToPackage{unicode}{hyperref}
\PassOptionsToPackage{hyphens}{url}
%
\documentclass[
]{article}
\usepackage{amsmath,amssymb}
\usepackage{iftex}
\ifPDFTeX
  \usepackage[T1]{fontenc}
  \usepackage[utf8]{inputenc}
  \usepackage{textcomp} % provide euro and other symbols
\else % if luatex or xetex
  \usepackage{unicode-math} % this also loads fontspec
  \defaultfontfeatures{Scale=MatchLowercase}
  \defaultfontfeatures[\rmfamily]{Ligatures=TeX,Scale=1}
\fi
\usepackage{lmodern}
\ifPDFTeX\else
  % xetex/luatex font selection
\fi
% Use upquote if available, for straight quotes in verbatim environments
\IfFileExists{upquote.sty}{\usepackage{upquote}}{}
\IfFileExists{microtype.sty}{% use microtype if available
  \usepackage[]{microtype}
  \UseMicrotypeSet[protrusion]{basicmath} % disable protrusion for tt fonts
}{}
\makeatletter
\@ifundefined{KOMAClassName}{% if non-KOMA class
  \IfFileExists{parskip.sty}{%
    \usepackage{parskip}
  }{% else
    \setlength{\parindent}{0pt}
    \setlength{\parskip}{6pt plus 2pt minus 1pt}}
}{% if KOMA class
  \KOMAoptions{parskip=half}}
\makeatother
\usepackage{xcolor}
\usepackage[margin=1in]{geometry}
\usepackage{color}
\usepackage{fancyvrb}
\newcommand{\VerbBar}{|}
\newcommand{\VERB}{\Verb[commandchars=\\\{\}]}
\DefineVerbatimEnvironment{Highlighting}{Verbatim}{commandchars=\\\{\}}
% Add ',fontsize=\small' for more characters per line
\usepackage{framed}
\definecolor{shadecolor}{RGB}{248,248,248}
\newenvironment{Shaded}{\begin{snugshade}}{\end{snugshade}}
\newcommand{\AlertTok}[1]{\textcolor[rgb]{0.94,0.16,0.16}{#1}}
\newcommand{\AnnotationTok}[1]{\textcolor[rgb]{0.56,0.35,0.01}{\textbf{\textit{#1}}}}
\newcommand{\AttributeTok}[1]{\textcolor[rgb]{0.13,0.29,0.53}{#1}}
\newcommand{\BaseNTok}[1]{\textcolor[rgb]{0.00,0.00,0.81}{#1}}
\newcommand{\BuiltInTok}[1]{#1}
\newcommand{\CharTok}[1]{\textcolor[rgb]{0.31,0.60,0.02}{#1}}
\newcommand{\CommentTok}[1]{\textcolor[rgb]{0.56,0.35,0.01}{\textit{#1}}}
\newcommand{\CommentVarTok}[1]{\textcolor[rgb]{0.56,0.35,0.01}{\textbf{\textit{#1}}}}
\newcommand{\ConstantTok}[1]{\textcolor[rgb]{0.56,0.35,0.01}{#1}}
\newcommand{\ControlFlowTok}[1]{\textcolor[rgb]{0.13,0.29,0.53}{\textbf{#1}}}
\newcommand{\DataTypeTok}[1]{\textcolor[rgb]{0.13,0.29,0.53}{#1}}
\newcommand{\DecValTok}[1]{\textcolor[rgb]{0.00,0.00,0.81}{#1}}
\newcommand{\DocumentationTok}[1]{\textcolor[rgb]{0.56,0.35,0.01}{\textbf{\textit{#1}}}}
\newcommand{\ErrorTok}[1]{\textcolor[rgb]{0.64,0.00,0.00}{\textbf{#1}}}
\newcommand{\ExtensionTok}[1]{#1}
\newcommand{\FloatTok}[1]{\textcolor[rgb]{0.00,0.00,0.81}{#1}}
\newcommand{\FunctionTok}[1]{\textcolor[rgb]{0.13,0.29,0.53}{\textbf{#1}}}
\newcommand{\ImportTok}[1]{#1}
\newcommand{\InformationTok}[1]{\textcolor[rgb]{0.56,0.35,0.01}{\textbf{\textit{#1}}}}
\newcommand{\KeywordTok}[1]{\textcolor[rgb]{0.13,0.29,0.53}{\textbf{#1}}}
\newcommand{\NormalTok}[1]{#1}
\newcommand{\OperatorTok}[1]{\textcolor[rgb]{0.81,0.36,0.00}{\textbf{#1}}}
\newcommand{\OtherTok}[1]{\textcolor[rgb]{0.56,0.35,0.01}{#1}}
\newcommand{\PreprocessorTok}[1]{\textcolor[rgb]{0.56,0.35,0.01}{\textit{#1}}}
\newcommand{\RegionMarkerTok}[1]{#1}
\newcommand{\SpecialCharTok}[1]{\textcolor[rgb]{0.81,0.36,0.00}{\textbf{#1}}}
\newcommand{\SpecialStringTok}[1]{\textcolor[rgb]{0.31,0.60,0.02}{#1}}
\newcommand{\StringTok}[1]{\textcolor[rgb]{0.31,0.60,0.02}{#1}}
\newcommand{\VariableTok}[1]{\textcolor[rgb]{0.00,0.00,0.00}{#1}}
\newcommand{\VerbatimStringTok}[1]{\textcolor[rgb]{0.31,0.60,0.02}{#1}}
\newcommand{\WarningTok}[1]{\textcolor[rgb]{0.56,0.35,0.01}{\textbf{\textit{#1}}}}
\usepackage{graphicx}
\makeatletter
\def\maxwidth{\ifdim\Gin@nat@width>\linewidth\linewidth\else\Gin@nat@width\fi}
\def\maxheight{\ifdim\Gin@nat@height>\textheight\textheight\else\Gin@nat@height\fi}
\makeatother
% Scale images if necessary, so that they will not overflow the page
% margins by default, and it is still possible to overwrite the defaults
% using explicit options in \includegraphics[width, height, ...]{}
\setkeys{Gin}{width=\maxwidth,height=\maxheight,keepaspectratio}
% Set default figure placement to htbp
\makeatletter
\def\fps@figure{htbp}
\makeatother
\setlength{\emergencystretch}{3em} % prevent overfull lines
\providecommand{\tightlist}{%
  \setlength{\itemsep}{0pt}\setlength{\parskip}{0pt}}
\setcounter{secnumdepth}{-\maxdimen} % remove section numbering
\ifLuaTeX
  \usepackage{selnolig}  % disable illegal ligatures
\fi
\IfFileExists{bookmark.sty}{\usepackage{bookmark}}{\usepackage{hyperref}}
\IfFileExists{xurl.sty}{\usepackage{xurl}}{} % add URL line breaks if available
\urlstyle{same}
\hypersetup{
  pdftitle={problem set 8 template},
  hidelinks,
  pdfcreator={LaTeX via pandoc}}

\title{problem set 8 template}
\author{}
\date{\vspace{-2.5em}}

\begin{document}
\maketitle

\hypertarget{question-1}{%
\section{Question 1}\label{question-1}}

\hypertarget{prelims}{%
\subsection{prelims}\label{prelims}}

\begin{Shaded}
\begin{Highlighting}[]
\FunctionTok{rm}\NormalTok{(}\AttributeTok{list=}\FunctionTok{ls}\NormalTok{())}
\FunctionTok{set.seed}\NormalTok{(}\DecValTok{123}\NormalTok{)}
\FunctionTok{library}\NormalTok{(rstan)}
\end{Highlighting}
\end{Shaded}

\begin{verbatim}
## Loading required package: StanHeaders
\end{verbatim}

\begin{verbatim}
## 
## rstan version 2.32.5 (Stan version 2.32.2)
\end{verbatim}

\begin{verbatim}
## For execution on a local, multicore CPU with excess RAM we recommend calling
## options(mc.cores = parallel::detectCores()).
## To avoid recompilation of unchanged Stan programs, we recommend calling
## rstan_options(auto_write = TRUE)
## For within-chain threading using `reduce_sum()` or `map_rect()` Stan functions,
## change `threads_per_chain` option:
## rstan_options(threads_per_chain = 1)
\end{verbatim}

\begin{Shaded}
\begin{Highlighting}[]
\FunctionTok{options}\NormalTok{(}\AttributeTok{mc.cores =}\NormalTok{ parallel}\SpecialCharTok{::}\FunctionTok{detectCores}\NormalTok{())}

\NormalTok{data }\OtherTok{=} \FunctionTok{read.csv}\NormalTok{(}\StringTok{"BattingAverage.csv"}\NormalTok{)}
\NormalTok{atbat }\OtherTok{=}\NormalTok{ data}\SpecialCharTok{$}\NormalTok{AtBats}
\NormalTok{position }\OtherTok{=}\NormalTok{ data}\SpecialCharTok{$}\NormalTok{PriPosNumber}
\NormalTok{hits }\OtherTok{=}\NormalTok{ data}\SpecialCharTok{$}\NormalTok{Hits}
\NormalTok{N }\OtherTok{=} \FunctionTok{length}\NormalTok{(hits)}
\end{Highlighting}
\end{Shaded}

\hypertarget{step-1}{%
\subsection{step 1}\label{step-1}}

\begin{Shaded}
\begin{Highlighting}[]
\NormalTok{fit }\OtherTok{=} \FunctionTok{stan}\NormalTok{(}\StringTok{"ps\_8.stan"}\NormalTok{, }\AttributeTok{iter=}\DecValTok{30000}\NormalTok{, }\AttributeTok{chains=}\DecValTok{4}\NormalTok{, }
           \AttributeTok{data=}\FunctionTok{list}\NormalTok{(}\AttributeTok{N =}\NormalTok{ N, }\AttributeTok{atbat =}\NormalTok{ atbat, }\AttributeTok{position =}\NormalTok{ position, }\AttributeTok{hits =}\NormalTok{ hits))}
\end{Highlighting}
\end{Shaded}

\begin{verbatim}
## Trying to compile a simple C file
\end{verbatim}

\begin{verbatim}
## Running /Library/Frameworks/R.framework/Resources/bin/R CMD SHLIB foo.c
## using C compiler: ‘Apple clang version 14.0.0 (clang-1400.0.29.202)’
## using SDK: ‘MacOSX13.1.sdk’
## clang -arch arm64 -I"/Library/Frameworks/R.framework/Resources/include" -DNDEBUG   -I"/Library/Frameworks/R.framework/Versions/4.3-arm64/Resources/library/Rcpp/include/"  -I"/Library/Frameworks/R.framework/Versions/4.3-arm64/Resources/library/RcppEigen/include/"  -I"/Library/Frameworks/R.framework/Versions/4.3-arm64/Resources/library/RcppEigen/include/unsupported"  -I"/Library/Frameworks/R.framework/Versions/4.3-arm64/Resources/library/BH/include" -I"/Library/Frameworks/R.framework/Versions/4.3-arm64/Resources/library/StanHeaders/include/src/"  -I"/Library/Frameworks/R.framework/Versions/4.3-arm64/Resources/library/StanHeaders/include/"  -I"/Library/Frameworks/R.framework/Versions/4.3-arm64/Resources/library/RcppParallel/include/"  -I"/Library/Frameworks/R.framework/Versions/4.3-arm64/Resources/library/rstan/include" -DEIGEN_NO_DEBUG  -DBOOST_DISABLE_ASSERTS  -DBOOST_PENDING_INTEGER_LOG2_HPP  -DSTAN_THREADS  -DUSE_STANC3 -DSTRICT_R_HEADERS  -DBOOST_PHOENIX_NO_VARIADIC_EXPRESSION  -D_HAS_AUTO_PTR_ETC=0  -include '/Library/Frameworks/R.framework/Versions/4.3-arm64/Resources/library/StanHeaders/include/stan/math/prim/fun/Eigen.hpp'  -D_REENTRANT -DRCPP_PARALLEL_USE_TBB=1   -I/opt/R/arm64/include    -fPIC  -falign-functions=64 -Wall -g -O2  -c foo.c -o foo.o
## In file included from <built-in>:1:
## In file included from /Library/Frameworks/R.framework/Versions/4.3-arm64/Resources/library/StanHeaders/include/stan/math/prim/fun/Eigen.hpp:22:
## In file included from /Library/Frameworks/R.framework/Versions/4.3-arm64/Resources/library/RcppEigen/include/Eigen/Dense:1:
## In file included from /Library/Frameworks/R.framework/Versions/4.3-arm64/Resources/library/RcppEigen/include/Eigen/Core:88:
## /Library/Frameworks/R.framework/Versions/4.3-arm64/Resources/library/RcppEigen/include/Eigen/src/Core/util/Macros.h:628:1: error: unknown type name 'namespace'
## namespace Eigen {
## ^
## /Library/Frameworks/R.framework/Versions/4.3-arm64/Resources/library/RcppEigen/include/Eigen/src/Core/util/Macros.h:628:16: error: expected ';' after top level declarator
## namespace Eigen {
##                ^
##                ;
## In file included from <built-in>:1:
## In file included from /Library/Frameworks/R.framework/Versions/4.3-arm64/Resources/library/StanHeaders/include/stan/math/prim/fun/Eigen.hpp:22:
## In file included from /Library/Frameworks/R.framework/Versions/4.3-arm64/Resources/library/RcppEigen/include/Eigen/Dense:1:
## /Library/Frameworks/R.framework/Versions/4.3-arm64/Resources/library/RcppEigen/include/Eigen/Core:96:10: fatal error: 'complex' file not found
## #include <complex>
##          ^~~~~~~~~
## 3 errors generated.
## make: *** [foo.o] Error 1
\end{verbatim}

\begin{verbatim}
## Warning: There were 1 chains where the estimated Bayesian Fraction of Missing Information was low. See
## https://mc-stan.org/misc/warnings.html#bfmi-low
\end{verbatim}

\begin{verbatim}
## Warning: Examine the pairs() plot to diagnose sampling problems
\end{verbatim}

\begin{Shaded}
\begin{Highlighting}[]
\FunctionTok{print}\NormalTok{(fit, }\AttributeTok{pars =} \FunctionTok{c}\NormalTok{(}\StringTok{"kc"}\NormalTok{, }\StringTok{"wc"}\NormalTok{, }\StringTok{"k"}\NormalTok{, }\StringTok{"w"}\NormalTok{, }\StringTok{"theta[1]"}\NormalTok{, }\StringTok{"theta[500]"}\NormalTok{, }
                    \StringTok{"theta[948]"}\NormalTok{), }\AttributeTok{probs =} \FunctionTok{c}\NormalTok{(}\FloatTok{0.05}\NormalTok{, }\FloatTok{0.5}\NormalTok{, }\FloatTok{0.95}\NormalTok{))}
\end{Highlighting}
\end{Shaded}

\begin{verbatim}
## Inference for Stan model: anon_model.
## 4 chains, each with iter=30000; warmup=15000; thin=1; 
## post-warmup draws per chain=15000, total post-warmup draws=60000.
## 
##              mean se_mean     sd     5%    50%    95%  n_eff Rhat
## kc[1]      105.22    1.14  45.64  52.90  95.11 191.24   1606    1
## kc[2]      170.13    0.36  46.98 105.45 163.78 256.48  17078    1
## kc[3]      237.53    0.48  69.35 144.09 227.23 364.80  20841    1
## kc[4]      256.25    0.60  81.63 147.20 244.10 408.33  18432    1
## kc[5]      210.61    0.49  66.84 121.73 200.46 334.35  18687    1
## kc[6]      240.41    0.62  83.78 130.96 226.84 397.39  18556    1
## kc[7]      213.91    0.52  62.87 130.19 204.81 329.11  14817    1
## kc[8]      238.37    0.53  76.14 136.12 227.06 378.75  20552    1
## kc[9]      399.47    0.96 133.81 220.28 379.03 647.58  19272    1
## wc[1]        0.12    0.00   0.01   0.11   0.12   0.13   4492    1
## wc[2]        0.24    0.00   0.01   0.23   0.24   0.25  39149    1
## wc[3]        0.25    0.00   0.00   0.24   0.25   0.26  45811    1
## wc[4]        0.25    0.00   0.01   0.24   0.25   0.26  44902    1
## wc[5]        0.26    0.00   0.01   0.25   0.26   0.26  42845    1
## wc[6]        0.25    0.00   0.01   0.24   0.25   0.26  43153    1
## wc[7]        0.25    0.00   0.00   0.24   0.25   0.26  35667    1
## wc[8]        0.26    0.00   0.01   0.25   0.26   0.27  47269    1
## wc[9]        0.26    0.00   0.00   0.25   0.26   0.27  42576    1
## k           59.54    0.16  31.89  18.11  54.08 119.82  42161    1
## w            0.23    0.00   0.02   0.19   0.23   0.26  20065    1
## theta[1]     0.13    0.00   0.03   0.08   0.13   0.19 103702    1
## theta[500]   0.24    0.00   0.03   0.19   0.24   0.30 114894    1
## theta[948]   0.27    0.00   0.01   0.24   0.27   0.29 115985    1
## 
## Samples were drawn using NUTS(diag_e) at Wed Feb 28 23:59:58 2024.
## For each parameter, n_eff is a crude measure of effective sample size,
## and Rhat is the potential scale reduction factor on split chains (at 
## convergence, Rhat=1).
\end{verbatim}

\hypertarget{step-2}{%
\subsection{Step 2}\label{step-2}}

\begin{Shaded}
\begin{Highlighting}[]
\CommentTok{\#Extract the Rhat statistic for the 968 parameters and plot them in a histogram}
\NormalTok{fit\_summary }\OtherTok{=} \FunctionTok{summary}\NormalTok{(fit)}\SpecialCharTok{$}\NormalTok{summary}
\NormalTok{rhats }\OtherTok{=}\NormalTok{ fit\_summary[}\DecValTok{1}\SpecialCharTok{:}\DecValTok{968}\NormalTok{,}\StringTok{"Rhat"}\NormalTok{]}
\NormalTok{ess }\OtherTok{=}\NormalTok{ fit\_summary[}\DecValTok{1}\SpecialCharTok{:}\DecValTok{968}\NormalTok{,}\StringTok{"n\_eff"}\NormalTok{]}
\FunctionTok{hist}\NormalTok{(rhats, }\AttributeTok{breaks =} \DecValTok{50}\NormalTok{)}
\end{Highlighting}
\end{Shaded}

\includegraphics{problem_set_8_template_files/figure-latex/unnamed-chunk-4-1.pdf}

\begin{Shaded}
\begin{Highlighting}[]
\CommentTok{\# Extract the ESS statistic for the 968 parameters and plot them in a histogram}
\FunctionTok{hist}\NormalTok{(ess)}
\end{Highlighting}
\end{Shaded}

\includegraphics{problem_set_8_template_files/figure-latex/unnamed-chunk-5-1.pdf}

\begin{Shaded}
\begin{Highlighting}[]
\FunctionTok{print}\NormalTok{(}\FunctionTok{min}\NormalTok{(ess))}
\end{Highlighting}
\end{Shaded}

\begin{verbatim}
## [1] 1606.347
\end{verbatim}

\begin{Shaded}
\begin{Highlighting}[]
\CommentTok{\#Identify the 10 parameters with the worse Rhat statistic and the 10 parameters}
\CommentTok{\#with the worse ESS statistic.}

\NormalTok{worst\_rhats }\OtherTok{=} \FunctionTok{sort}\NormalTok{(rhats)[}\DecValTok{1}\SpecialCharTok{:}\DecValTok{10}\NormalTok{]}
\NormalTok{worst\_esss }\OtherTok{=} \FunctionTok{sort}\NormalTok{(ess)[}\DecValTok{1}\SpecialCharTok{:}\DecValTok{10}\NormalTok{]}

\FunctionTok{print}\NormalTok{(}\FunctionTok{data.frame}\NormalTok{(}\AttributeTok{Parameter =}\NormalTok{ worst\_rhats))}
\end{Highlighting}
\end{Shaded}

\begin{verbatim}
##            Parameter
## theta[582] 0.9999375
## theta[922] 0.9999380
## theta[762] 0.9999380
## theta[751] 0.9999385
## theta[602] 0.9999388
## theta[328] 0.9999395
## theta[876] 0.9999395
## theta[607] 0.9999396
## theta[275] 0.9999399
## theta[188] 0.9999401
\end{verbatim}

\begin{Shaded}
\begin{Highlighting}[]
\FunctionTok{print}\NormalTok{(}\FunctionTok{data.frame}\NormalTok{(}\AttributeTok{Parameter =}\NormalTok{ worst\_esss))}
\end{Highlighting}
\end{Shaded}

\begin{verbatim}
##            Parameter
## kc[1]       1606.347
## wc[1]       4492.336
## theta[494] 10542.770
## theta[358] 12184.444
## theta[839] 13520.697
## theta[102] 14386.723
## kc[7]      14817.002
## kc[2]      17078.407
## kc[4]      18431.636
## kc[6]      18556.276
\end{verbatim}

\begin{Shaded}
\begin{Highlighting}[]
\CommentTok{\#Inspect the trace plots for both hyperparameters, for the field position mode }
\CommentTok{\#and concentrations, and for the theta of the 1st, 500th, and 948th player in }
\CommentTok{\#the dataset.}
\FunctionTok{traceplot}\NormalTok{(fit, }\AttributeTok{inc\_warmup =} \ConstantTok{FALSE}\NormalTok{, }\AttributeTok{nrow =} \DecValTok{5}\NormalTok{, }\AttributeTok{window =} \FunctionTok{c}\NormalTok{(}\DecValTok{25000}\NormalTok{, }\DecValTok{30000}\NormalTok{), }
          \AttributeTok{pars =} \FunctionTok{c}\NormalTok{(}\StringTok{"k"}\NormalTok{, }\StringTok{"w"}\NormalTok{, }\StringTok{"theta[1]"}\NormalTok{, }\StringTok{"theta[500]"}\NormalTok{, }\StringTok{"theta[948]"}\NormalTok{))}
\end{Highlighting}
\end{Shaded}

\includegraphics{problem_set_8_template_files/figure-latex/unnamed-chunk-7-1.pdf}

\begin{Shaded}
\begin{Highlighting}[]
\FunctionTok{traceplot}\NormalTok{(fit, }\AttributeTok{inc\_warmup =} \ConstantTok{FALSE}\NormalTok{, }\AttributeTok{window =} \FunctionTok{c}\NormalTok{(}\DecValTok{25000}\NormalTok{, }\DecValTok{30000}\NormalTok{), }
          \AttributeTok{pars =} \FunctionTok{c}\NormalTok{(}\StringTok{"kc"}\NormalTok{, }\StringTok{"wc"}\NormalTok{))}
\end{Highlighting}
\end{Shaded}

\includegraphics{problem_set_8_template_files/figure-latex/unnamed-chunk-8-1.pdf}
Everything seems to converge. \#\# step 3

\begin{Shaded}
\begin{Highlighting}[]
\NormalTok{samples }\OtherTok{=} \FunctionTok{extract}\NormalTok{(fit)}
\NormalTok{w\_catcher }\OtherTok{=}\NormalTok{ samples}\SpecialCharTok{$}\NormalTok{wc[, }\DecValTok{2}\NormalTok{]}
\NormalTok{w\_pitcher }\OtherTok{=}\NormalTok{ samples}\SpecialCharTok{$}\NormalTok{wc[, }\DecValTok{1}\NormalTok{]}
\NormalTok{difference }\OtherTok{=}\NormalTok{ w\_catcher }\SpecialCharTok{{-}}\NormalTok{ w\_pitcher}
\FunctionTok{hist}\NormalTok{(difference, }\AttributeTok{breaks =} \DecValTok{40}\NormalTok{, }\AttributeTok{main=}\StringTok{"Posterior Differences Between Catchers and Pitchers"}\NormalTok{)}
\end{Highlighting}
\end{Shaded}

\includegraphics{problem_set_8_template_files/figure-latex/unnamed-chunk-9-1.pdf}

\begin{Shaded}
\begin{Highlighting}[]
\FunctionTok{library}\NormalTok{(bayestestR)}
\FunctionTok{hdi}\NormalTok{(difference)}
\end{Highlighting}
\end{Shaded}

\begin{verbatim}
## 95% HDI: [0.10, 0.13]
\end{verbatim}

\begin{Shaded}
\begin{Highlighting}[]
\FunctionTok{paste}\NormalTok{(}\StringTok{"|P(w\_catcher \textgreater{} w\_pitcher): "}\NormalTok{, }\FunctionTok{mean}\NormalTok{(difference }\SpecialCharTok{\textgreater{}} \DecValTok{0}\NormalTok{))}
\end{Highlighting}
\end{Shaded}

\begin{verbatim}
## [1] "|P(w_catcher > w_pitcher):  1"
\end{verbatim}

\hypertarget{step-4}{%
\subsection{Step 4}\label{step-4}}

\begin{Shaded}
\begin{Highlighting}[]
\NormalTok{w\_catcher }\OtherTok{=}\NormalTok{ samples}\SpecialCharTok{$}\NormalTok{wc[, }\DecValTok{2}\NormalTok{]}
\NormalTok{w\_first\_base }\OtherTok{=}\NormalTok{ samples}\SpecialCharTok{$}\NormalTok{wc[, }\DecValTok{3}\NormalTok{]}
\NormalTok{diff2 }\OtherTok{=}\NormalTok{ w\_catcher }\SpecialCharTok{{-}}\NormalTok{ w\_first\_base}
\FunctionTok{hist}\NormalTok{(diff2, }\AttributeTok{breaks =} \DecValTok{40}\NormalTok{, }\AttributeTok{main=}\StringTok{"Posterior Differences Between Catchers and First Base"}\NormalTok{)}
\end{Highlighting}
\end{Shaded}

\includegraphics{problem_set_8_template_files/figure-latex/unnamed-chunk-11-1.pdf}

\begin{Shaded}
\begin{Highlighting}[]
\FunctionTok{hdi}\NormalTok{(diff2)}
\end{Highlighting}
\end{Shaded}

\begin{verbatim}
## 95% HDI: [-0.03,  0.00]
\end{verbatim}

\begin{Shaded}
\begin{Highlighting}[]
\FunctionTok{paste}\NormalTok{(}\StringTok{"|P(w\_catcher \textgreater{} w\_first\_base): "}\NormalTok{, }\FunctionTok{mean}\NormalTok{(diff2 }\SpecialCharTok{\textgreater{}} \DecValTok{0}\NormalTok{))}
\end{Highlighting}
\end{Shaded}

\begin{verbatim}
## [1] "|P(w_catcher > w_first_base):  0.0220333333333333"
\end{verbatim}

\hypertarget{step-5}{%
\subsection{step 5}\label{step-5}}

\begin{Shaded}
\begin{Highlighting}[]
\FunctionTok{hist}\NormalTok{(samples}\SpecialCharTok{$}\NormalTok{w, }\AttributeTok{freq =} \ConstantTok{FALSE}\NormalTok{, }\AttributeTok{xlim =} \FunctionTok{c}\NormalTok{(}\FunctionTok{min}\NormalTok{(samples}\SpecialCharTok{$}\NormalTok{wc), }\FunctionTok{max}\NormalTok{(samples}\SpecialCharTok{$}\NormalTok{wc)), }\AttributeTok{ylim=}\FunctionTok{c}\NormalTok{(}\DecValTok{0}\NormalTok{, }\DecValTok{80}\NormalTok{),}
     \AttributeTok{main =} \StringTok{"Posterior Distribution Of Batting Averages"}\NormalTok{,}
     \AttributeTok{xlab =} \StringTok{"Batting Average"}\NormalTok{, }\AttributeTok{breaks=}\DecValTok{50}\NormalTok{)}

\FunctionTok{lines}\NormalTok{(}\FunctionTok{density}\NormalTok{(samples}\SpecialCharTok{$}\NormalTok{w), }\AttributeTok{lwd =} \DecValTok{2}\NormalTok{, }\AttributeTok{col =} \StringTok{"blue"}\NormalTok{)}
\NormalTok{colors }\OtherTok{=} \FunctionTok{rainbow}\NormalTok{(}\DecValTok{9}\NormalTok{)}
\ControlFlowTok{for}\NormalTok{ (i }\ControlFlowTok{in} \DecValTok{1}\SpecialCharTok{:}\DecValTok{9}\NormalTok{) \{}
  \FunctionTok{lines}\NormalTok{(}\FunctionTok{density}\NormalTok{(samples}\SpecialCharTok{$}\NormalTok{wc[,i]), }\AttributeTok{lwd =} \DecValTok{2}\NormalTok{, }\AttributeTok{col =}\NormalTok{ colors[i])}
\NormalTok{\}}

\FunctionTok{legend}\NormalTok{(}\StringTok{"topleft"}\NormalTok{, }\AttributeTok{legend =} \FunctionTok{c}\NormalTok{(}\StringTok{"w"}\NormalTok{, }\FunctionTok{paste}\NormalTok{(}\StringTok{"wc"}\NormalTok{, }\DecValTok{1}\SpecialCharTok{:}\DecValTok{9}\NormalTok{, }\AttributeTok{sep =} \StringTok{""}\NormalTok{)), }\AttributeTok{fill =} \FunctionTok{c}\NormalTok{(}\StringTok{"blue"}\NormalTok{, colors), }\AttributeTok{cex =} \FloatTok{0.8}\NormalTok{)}
\end{Highlighting}
\end{Shaded}

\includegraphics{problem_set_8_template_files/figure-latex/unnamed-chunk-13-1.pdf}

This is because w is only informed by nine parameters wc(1:9) whereas
each wc is informed by way more data points relating to the player's
positions. Therefore, as wc has more data points informing it, it has a
tighter posterior compared to w which has 9.

\hypertarget{step-6}{%
\subsection{step 6}\label{step-6}}

\begin{Shaded}
\begin{Highlighting}[]
\FunctionTok{par}\NormalTok{(}\AttributeTok{mfrow=}\FunctionTok{c}\NormalTok{(}\DecValTok{3}\NormalTok{,}\DecValTok{3}\NormalTok{))}

\ControlFlowTok{for}\NormalTok{ (i }\ControlFlowTok{in} \DecValTok{1}\SpecialCharTok{:}\DecValTok{9}\NormalTok{) \{}
\NormalTok{  smpw }\OtherTok{=} \FunctionTok{sample}\NormalTok{(samples}\SpecialCharTok{$}\NormalTok{wc[,i], }\DecValTok{10000}\NormalTok{)}
\NormalTok{  smpk }\OtherTok{=} \FunctionTok{sample}\NormalTok{(samples}\SpecialCharTok{$}\NormalTok{kc[,i], }\DecValTok{10000}\NormalTok{)}
\NormalTok{  hit\_preds }\OtherTok{=} \FunctionTok{rbeta}\NormalTok{(}\DecValTok{1000}\NormalTok{, smpw }\SpecialCharTok{*}\NormalTok{ smpk, (}\DecValTok{1} \SpecialCharTok{{-}}\NormalTok{ smpw) }\SpecialCharTok{*}\NormalTok{ smpk)}
\NormalTok{  hits\_ratio }\OtherTok{=}\NormalTok{ hits[position }\SpecialCharTok{==}\NormalTok{ i] }\SpecialCharTok{/}\NormalTok{ atbat[position }\SpecialCharTok{==}\NormalTok{ i]}
  
  \CommentTok{\# Plot the histogram of observed hit ratios}
  \FunctionTok{hist}\NormalTok{(hits\_ratio, }\AttributeTok{probability =} \ConstantTok{TRUE}\NormalTok{, }\AttributeTok{main =} \FunctionTok{paste}\NormalTok{(}\StringTok{"Position"}\NormalTok{, i), }
       \AttributeTok{xlab =} \StringTok{"Hit Ratio"}\NormalTok{, }\AttributeTok{ylab =} \StringTok{"Density"}\NormalTok{, }\AttributeTok{xlim =} \FunctionTok{range}\NormalTok{(}\FunctionTok{c}\NormalTok{(hits\_ratio, hit\_preds)),}
       \AttributeTok{ylim=}\FunctionTok{c}\NormalTok{(}\DecValTok{0}\NormalTok{, }\DecValTok{18}\NormalTok{), }\AttributeTok{breaks=}\DecValTok{20}\NormalTok{)}
  
  \FunctionTok{lines}\NormalTok{(}\FunctionTok{density}\NormalTok{(hit\_preds), }\AttributeTok{col =} \StringTok{"blue"}\NormalTok{)}
\NormalTok{\}}
\end{Highlighting}
\end{Shaded}

\includegraphics{problem_set_8_template_files/figure-latex/unnamed-chunk-14-1.pdf}

\hypertarget{step-7}{%
\subsection{step 7}\label{step-7}}

No as the pitcher is not really related to the others as pitcher is
clearly an outlier here.

\hypertarget{question-2}{%
\section{Question 2}\label{question-2}}

\hypertarget{step-1-1}{%
\subsection{step 1}\label{step-1-1}}

\begin{Shaded}
\begin{Highlighting}[]
\NormalTok{fit\_2 }\OtherTok{=} \FunctionTok{stan}\NormalTok{(}\StringTok{"ps\_8\_2.stan"}\NormalTok{, }\AttributeTok{iter=}\DecValTok{25000}\NormalTok{, }\AttributeTok{chains=}\DecValTok{4}\NormalTok{, }
           \AttributeTok{data=}\FunctionTok{list}\NormalTok{(}\AttributeTok{N =}\NormalTok{ N, }\AttributeTok{atbat =}\NormalTok{ atbat, }\AttributeTok{position =}\NormalTok{ position, }\AttributeTok{hits =}\NormalTok{ hits))}
\end{Highlighting}
\end{Shaded}

\begin{verbatim}
## Trying to compile a simple C file
\end{verbatim}

\begin{verbatim}
## Running /Library/Frameworks/R.framework/Resources/bin/R CMD SHLIB foo.c
## using C compiler: ‘Apple clang version 14.0.0 (clang-1400.0.29.202)’
## using SDK: ‘MacOSX13.1.sdk’
## clang -arch arm64 -I"/Library/Frameworks/R.framework/Resources/include" -DNDEBUG   -I"/Library/Frameworks/R.framework/Versions/4.3-arm64/Resources/library/Rcpp/include/"  -I"/Library/Frameworks/R.framework/Versions/4.3-arm64/Resources/library/RcppEigen/include/"  -I"/Library/Frameworks/R.framework/Versions/4.3-arm64/Resources/library/RcppEigen/include/unsupported"  -I"/Library/Frameworks/R.framework/Versions/4.3-arm64/Resources/library/BH/include" -I"/Library/Frameworks/R.framework/Versions/4.3-arm64/Resources/library/StanHeaders/include/src/"  -I"/Library/Frameworks/R.framework/Versions/4.3-arm64/Resources/library/StanHeaders/include/"  -I"/Library/Frameworks/R.framework/Versions/4.3-arm64/Resources/library/RcppParallel/include/"  -I"/Library/Frameworks/R.framework/Versions/4.3-arm64/Resources/library/rstan/include" -DEIGEN_NO_DEBUG  -DBOOST_DISABLE_ASSERTS  -DBOOST_PENDING_INTEGER_LOG2_HPP  -DSTAN_THREADS  -DUSE_STANC3 -DSTRICT_R_HEADERS  -DBOOST_PHOENIX_NO_VARIADIC_EXPRESSION  -D_HAS_AUTO_PTR_ETC=0  -include '/Library/Frameworks/R.framework/Versions/4.3-arm64/Resources/library/StanHeaders/include/stan/math/prim/fun/Eigen.hpp'  -D_REENTRANT -DRCPP_PARALLEL_USE_TBB=1   -I/opt/R/arm64/include    -fPIC  -falign-functions=64 -Wall -g -O2  -c foo.c -o foo.o
## In file included from <built-in>:1:
## In file included from /Library/Frameworks/R.framework/Versions/4.3-arm64/Resources/library/StanHeaders/include/stan/math/prim/fun/Eigen.hpp:22:
## In file included from /Library/Frameworks/R.framework/Versions/4.3-arm64/Resources/library/RcppEigen/include/Eigen/Dense:1:
## In file included from /Library/Frameworks/R.framework/Versions/4.3-arm64/Resources/library/RcppEigen/include/Eigen/Core:88:
## /Library/Frameworks/R.framework/Versions/4.3-arm64/Resources/library/RcppEigen/include/Eigen/src/Core/util/Macros.h:628:1: error: unknown type name 'namespace'
## namespace Eigen {
## ^
## /Library/Frameworks/R.framework/Versions/4.3-arm64/Resources/library/RcppEigen/include/Eigen/src/Core/util/Macros.h:628:16: error: expected ';' after top level declarator
## namespace Eigen {
##                ^
##                ;
## In file included from <built-in>:1:
## In file included from /Library/Frameworks/R.framework/Versions/4.3-arm64/Resources/library/StanHeaders/include/stan/math/prim/fun/Eigen.hpp:22:
## In file included from /Library/Frameworks/R.framework/Versions/4.3-arm64/Resources/library/RcppEigen/include/Eigen/Dense:1:
## /Library/Frameworks/R.framework/Versions/4.3-arm64/Resources/library/RcppEigen/include/Eigen/Core:96:10: fatal error: 'complex' file not found
## #include <complex>
##          ^~~~~~~~~
## 3 errors generated.
## make: *** [foo.o] Error 1
\end{verbatim}

\begin{verbatim}
## Warning: There were 1 transitions after warmup that exceeded the maximum treedepth. Increase max_treedepth above 10. See
## https://mc-stan.org/misc/warnings.html#maximum-treedepth-exceeded
\end{verbatim}

\begin{verbatim}
## Warning: Examine the pairs() plot to diagnose sampling problems
\end{verbatim}

\hypertarget{step-2-1}{%
\subsection{step 2}\label{step-2-1}}

\begin{Shaded}
\begin{Highlighting}[]
\NormalTok{samples\_2 }\OtherTok{=} \FunctionTok{extract}\NormalTok{(fit\_2)}
\FunctionTok{hist}\NormalTok{(samples\_2}\SpecialCharTok{$}\NormalTok{w, }\AttributeTok{freq =} \ConstantTok{FALSE}\NormalTok{, }\AttributeTok{xlim =} \FunctionTok{c}\NormalTok{(}\FunctionTok{min}\NormalTok{(samples\_2}\SpecialCharTok{$}\NormalTok{wc), }\FunctionTok{max}\NormalTok{(samples\_2}\SpecialCharTok{$}\NormalTok{wc)), }\AttributeTok{ylim=}\FunctionTok{c}\NormalTok{(}\DecValTok{0}\NormalTok{, }\DecValTok{90}\NormalTok{),}
     \AttributeTok{main =} \StringTok{"Posterior Distribution Of Batting Averages"}\NormalTok{,}
     \AttributeTok{xlab =} \StringTok{"Batting Average"}\NormalTok{, }\AttributeTok{breaks=}\DecValTok{50}\NormalTok{)}

\FunctionTok{lines}\NormalTok{(}\FunctionTok{density}\NormalTok{(samples\_2}\SpecialCharTok{$}\NormalTok{w), }\AttributeTok{lwd =} \DecValTok{2}\NormalTok{, }\AttributeTok{col =} \StringTok{"blue"}\NormalTok{)}
\NormalTok{colors }\OtherTok{=} \FunctionTok{rainbow}\NormalTok{(}\DecValTok{9}\NormalTok{)}
\ControlFlowTok{for}\NormalTok{ (i }\ControlFlowTok{in} \DecValTok{1}\SpecialCharTok{:}\DecValTok{9}\NormalTok{) \{}
  \FunctionTok{lines}\NormalTok{(}\FunctionTok{density}\NormalTok{(samples\_2}\SpecialCharTok{$}\NormalTok{wc[,i]), }\AttributeTok{lwd =} \DecValTok{2}\NormalTok{, }\AttributeTok{col =}\NormalTok{ colors[i])}
\NormalTok{\}}

\FunctionTok{legend}\NormalTok{(}\StringTok{"topleft"}\NormalTok{, }\AttributeTok{legend =} \FunctionTok{c}\NormalTok{(}\StringTok{"w"}\NormalTok{, }\FunctionTok{paste}\NormalTok{(}\StringTok{"wc"}\NormalTok{, }\DecValTok{1}\SpecialCharTok{:}\DecValTok{9}\NormalTok{, }\AttributeTok{sep =} \StringTok{""}\NormalTok{)), }
       \AttributeTok{fill =} \FunctionTok{c}\NormalTok{(}\StringTok{"blue"}\NormalTok{, colors), }\AttributeTok{cex =} \FloatTok{0.8}\NormalTok{)}
\end{Highlighting}
\end{Shaded}

\includegraphics{problem_set_8_template_files/figure-latex/unnamed-chunk-16-1.pdf}

\hypertarget{step-3}{%
\subsection{step 3}\label{step-3}}

\begin{Shaded}
\begin{Highlighting}[]
\FunctionTok{par}\NormalTok{(}\AttributeTok{mfrow=}\FunctionTok{c}\NormalTok{(}\DecValTok{3}\NormalTok{,}\DecValTok{3}\NormalTok{))}

\ControlFlowTok{for}\NormalTok{ (i }\ControlFlowTok{in} \DecValTok{1}\SpecialCharTok{:}\DecValTok{8}\NormalTok{) \{}
\NormalTok{  smpw\_2 }\OtherTok{=} \FunctionTok{sample}\NormalTok{(samples\_2}\SpecialCharTok{$}\NormalTok{wc[,i], }\DecValTok{10000}\NormalTok{)}
\NormalTok{  smpk\_2 }\OtherTok{=} \FunctionTok{sample}\NormalTok{(samples\_2}\SpecialCharTok{$}\NormalTok{kc[,i], }\DecValTok{10000}\NormalTok{)}
\NormalTok{  hit\_preds\_2 }\OtherTok{=} \FunctionTok{rbeta}\NormalTok{(}\DecValTok{1000}\NormalTok{, smpw\_2 }\SpecialCharTok{*}\NormalTok{ smpk\_2, (}\DecValTok{1} \SpecialCharTok{{-}}\NormalTok{ smpw\_2) }\SpecialCharTok{*}\NormalTok{ smpk\_2)}
\NormalTok{  hits\_ratio\_2 }\OtherTok{=}\NormalTok{ hits[position }\SpecialCharTok{==}\NormalTok{ i] }\SpecialCharTok{/}\NormalTok{ atbat[position }\SpecialCharTok{==}\NormalTok{ i]}
  
  \CommentTok{\# Plot the histogram of observed hit ratios}
  \FunctionTok{hist}\NormalTok{(hits\_ratio\_2, }\AttributeTok{probability =} \ConstantTok{TRUE}\NormalTok{, }\AttributeTok{main =} \FunctionTok{paste}\NormalTok{(}\StringTok{"Position"}\NormalTok{, i), }
       \AttributeTok{xlab =} \StringTok{"Hit Ratio"}\NormalTok{, }\AttributeTok{ylab =} \StringTok{"Density"}\NormalTok{, }\AttributeTok{xlim =} \FunctionTok{range}\NormalTok{(}\FunctionTok{c}\NormalTok{(hits\_ratio\_2, hit\_preds\_2)),}
       \AttributeTok{ylim=}\FunctionTok{c}\NormalTok{(}\DecValTok{0}\NormalTok{, }\DecValTok{18}\NormalTok{), }\AttributeTok{breaks=}\DecValTok{20}\NormalTok{)}
  
  \FunctionTok{lines}\NormalTok{(}\FunctionTok{density}\NormalTok{(hit\_preds\_2), }\AttributeTok{col =} \StringTok{"blue"}\NormalTok{)}
\NormalTok{\}}
\end{Highlighting}
\end{Shaded}

\includegraphics{problem_set_8_template_files/figure-latex/unnamed-chunk-17-1.pdf}

\hypertarget{step-4-1}{%
\subsection{step 4}\label{step-4-1}}

The hyper parameter w already has very few data points influencing it
hence removing one of these data points would lead to a drastic change
in w. Furthermore, the pitcher had the most variance compared to other
position's batting average as the position has the furthest batting
average compared to the rest therefore removing it would lead to this
tighter posterior for w as all of the other positions were pretty close
in batting averages as seen in step 2.

\end{document}

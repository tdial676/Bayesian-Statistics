% Options for packages loaded elsewhere
\PassOptionsToPackage{unicode}{hyperref}
\PassOptionsToPackage{hyphens}{url}
%
\documentclass[
]{article}
\usepackage{amsmath,amssymb}
\usepackage{iftex}
\ifPDFTeX
  \usepackage[T1]{fontenc}
  \usepackage[utf8]{inputenc}
  \usepackage{textcomp} % provide euro and other symbols
\else % if luatex or xetex
  \usepackage{unicode-math} % this also loads fontspec
  \defaultfontfeatures{Scale=MatchLowercase}
  \defaultfontfeatures[\rmfamily]{Ligatures=TeX,Scale=1}
\fi
\usepackage{lmodern}
\ifPDFTeX\else
  % xetex/luatex font selection
\fi
% Use upquote if available, for straight quotes in verbatim environments
\IfFileExists{upquote.sty}{\usepackage{upquote}}{}
\IfFileExists{microtype.sty}{% use microtype if available
  \usepackage[]{microtype}
  \UseMicrotypeSet[protrusion]{basicmath} % disable protrusion for tt fonts
}{}
\makeatletter
\@ifundefined{KOMAClassName}{% if non-KOMA class
  \IfFileExists{parskip.sty}{%
    \usepackage{parskip}
  }{% else
    \setlength{\parindent}{0pt}
    \setlength{\parskip}{6pt plus 2pt minus 1pt}}
}{% if KOMA class
  \KOMAoptions{parskip=half}}
\makeatother
\usepackage{xcolor}
\usepackage[margin=1in]{geometry}
\usepackage{color}
\usepackage{fancyvrb}
\newcommand{\VerbBar}{|}
\newcommand{\VERB}{\Verb[commandchars=\\\{\}]}
\DefineVerbatimEnvironment{Highlighting}{Verbatim}{commandchars=\\\{\}}
% Add ',fontsize=\small' for more characters per line
\usepackage{framed}
\definecolor{shadecolor}{RGB}{248,248,248}
\newenvironment{Shaded}{\begin{snugshade}}{\end{snugshade}}
\newcommand{\AlertTok}[1]{\textcolor[rgb]{0.94,0.16,0.16}{#1}}
\newcommand{\AnnotationTok}[1]{\textcolor[rgb]{0.56,0.35,0.01}{\textbf{\textit{#1}}}}
\newcommand{\AttributeTok}[1]{\textcolor[rgb]{0.13,0.29,0.53}{#1}}
\newcommand{\BaseNTok}[1]{\textcolor[rgb]{0.00,0.00,0.81}{#1}}
\newcommand{\BuiltInTok}[1]{#1}
\newcommand{\CharTok}[1]{\textcolor[rgb]{0.31,0.60,0.02}{#1}}
\newcommand{\CommentTok}[1]{\textcolor[rgb]{0.56,0.35,0.01}{\textit{#1}}}
\newcommand{\CommentVarTok}[1]{\textcolor[rgb]{0.56,0.35,0.01}{\textbf{\textit{#1}}}}
\newcommand{\ConstantTok}[1]{\textcolor[rgb]{0.56,0.35,0.01}{#1}}
\newcommand{\ControlFlowTok}[1]{\textcolor[rgb]{0.13,0.29,0.53}{\textbf{#1}}}
\newcommand{\DataTypeTok}[1]{\textcolor[rgb]{0.13,0.29,0.53}{#1}}
\newcommand{\DecValTok}[1]{\textcolor[rgb]{0.00,0.00,0.81}{#1}}
\newcommand{\DocumentationTok}[1]{\textcolor[rgb]{0.56,0.35,0.01}{\textbf{\textit{#1}}}}
\newcommand{\ErrorTok}[1]{\textcolor[rgb]{0.64,0.00,0.00}{\textbf{#1}}}
\newcommand{\ExtensionTok}[1]{#1}
\newcommand{\FloatTok}[1]{\textcolor[rgb]{0.00,0.00,0.81}{#1}}
\newcommand{\FunctionTok}[1]{\textcolor[rgb]{0.13,0.29,0.53}{\textbf{#1}}}
\newcommand{\ImportTok}[1]{#1}
\newcommand{\InformationTok}[1]{\textcolor[rgb]{0.56,0.35,0.01}{\textbf{\textit{#1}}}}
\newcommand{\KeywordTok}[1]{\textcolor[rgb]{0.13,0.29,0.53}{\textbf{#1}}}
\newcommand{\NormalTok}[1]{#1}
\newcommand{\OperatorTok}[1]{\textcolor[rgb]{0.81,0.36,0.00}{\textbf{#1}}}
\newcommand{\OtherTok}[1]{\textcolor[rgb]{0.56,0.35,0.01}{#1}}
\newcommand{\PreprocessorTok}[1]{\textcolor[rgb]{0.56,0.35,0.01}{\textit{#1}}}
\newcommand{\RegionMarkerTok}[1]{#1}
\newcommand{\SpecialCharTok}[1]{\textcolor[rgb]{0.81,0.36,0.00}{\textbf{#1}}}
\newcommand{\SpecialStringTok}[1]{\textcolor[rgb]{0.31,0.60,0.02}{#1}}
\newcommand{\StringTok}[1]{\textcolor[rgb]{0.31,0.60,0.02}{#1}}
\newcommand{\VariableTok}[1]{\textcolor[rgb]{0.00,0.00,0.00}{#1}}
\newcommand{\VerbatimStringTok}[1]{\textcolor[rgb]{0.31,0.60,0.02}{#1}}
\newcommand{\WarningTok}[1]{\textcolor[rgb]{0.56,0.35,0.01}{\textbf{\textit{#1}}}}
\usepackage{graphicx}
\makeatletter
\def\maxwidth{\ifdim\Gin@nat@width>\linewidth\linewidth\else\Gin@nat@width\fi}
\def\maxheight{\ifdim\Gin@nat@height>\textheight\textheight\else\Gin@nat@height\fi}
\makeatother
% Scale images if necessary, so that they will not overflow the page
% margins by default, and it is still possible to overwrite the defaults
% using explicit options in \includegraphics[width, height, ...]{}
\setkeys{Gin}{width=\maxwidth,height=\maxheight,keepaspectratio}
% Set default figure placement to htbp
\makeatletter
\def\fps@figure{htbp}
\makeatother
\setlength{\emergencystretch}{3em} % prevent overfull lines
\providecommand{\tightlist}{%
  \setlength{\itemsep}{0pt}\setlength{\parskip}{0pt}}
\setcounter{secnumdepth}{-\maxdimen} % remove section numbering
\ifLuaTeX
  \usepackage{selnolig}  % disable illegal ligatures
\fi
\IfFileExists{bookmark.sty}{\usepackage{bookmark}}{\usepackage{hyperref}}
\IfFileExists{xurl.sty}{\usepackage{xurl}}{} % add URL line breaks if available
\urlstyle{same}
\hypersetup{
  pdftitle={Ec/ACM/CS 112. Problem set 2. Solutions},
  hidelinks,
  pdfcreator={LaTeX via pandoc}}

\title{Ec/ACM/CS 112. Problem set 2. Solutions}
\author{}
\date{\vspace{-2.5em}}

\begin{document}
\maketitle

\hypertarget{preliminaries}{%
\section{Preliminaries}\label{preliminaries}}

\begin{Shaded}
\begin{Highlighting}[]
\FunctionTok{rm}\NormalTok{(}\AttributeTok{list =} \FunctionTok{ls}\NormalTok{())}
\FunctionTok{set.seed}\NormalTok{(}\DecValTok{123}\NormalTok{)}
\end{Highlighting}
\end{Shaded}

\hypertarget{part-1}{%
\section{PART 1}\label{part-1}}

\#Define Params and load data

\begin{Shaded}
\begin{Highlighting}[]
\NormalTok{data }\OtherTok{=} \FunctionTok{read.csv}\NormalTok{(}\StringTok{"PS2\_data.csv"}\NormalTok{)}
\NormalTok{aPrior }\OtherTok{=} \DecValTok{5}
\NormalTok{bPrior }\OtherTok{=} \DecValTok{5}
\NormalTok{nGridPoints }\OtherTok{=} \DecValTok{100}
\NormalTok{pGrid }\OtherTok{=} \FunctionTok{seq}\NormalTok{(}\AttributeTok{from =} \DecValTok{0}\NormalTok{, }\AttributeTok{to =} \DecValTok{1}\NormalTok{, }\AttributeTok{length.out =}\NormalTok{ nGridPoints)}
\NormalTok{gridSize }\OtherTok{=} \DecValTok{1} \SpecialCharTok{/}\NormalTok{ nGridPoints}
\end{Highlighting}
\end{Shaded}

\#Helper Functions

\begin{Shaded}
\begin{Highlighting}[]
\CommentTok{\#compute prior matrix}
\NormalTok{computeJointPrior }\OtherTok{\textless{}{-}}\ControlFlowTok{function}\NormalTok{()\{}
\NormalTok{  prior }\OtherTok{=} \FunctionTok{dbeta}\NormalTok{(}\AttributeTok{x =}\NormalTok{ pGrid, }\AttributeTok{shape1 =}\NormalTok{ aPrior, }\AttributeTok{shape2 =}\NormalTok{ bPrior)}
  \FunctionTok{return}\NormalTok{(}\FunctionTok{outer}\NormalTok{(prior, prior, }\StringTok{"*"}\NormalTok{))}
\NormalTok{\}}

\CommentTok{\#compute posterior matrix}
\NormalTok{computeJointPosterior }\OtherTok{\textless{}{-}} \ControlFlowTok{function}\NormalTok{(ball1\_success, ball2\_success, n1, n2, priorM) \{}
\NormalTok{  postM }\OtherTok{=} \FunctionTok{matrix}\NormalTok{(}\FunctionTok{rep}\NormalTok{(}\DecValTok{1}\NormalTok{, nGridPoints }\SpecialCharTok{\^{}} \DecValTok{2}\NormalTok{),}
                  \AttributeTok{nrow =}\NormalTok{ nGridPoints,}
                  \AttributeTok{ncol =}\NormalTok{ nGridPoints,}
                  \AttributeTok{byrow =} \ConstantTok{TRUE}\NormalTok{)}
  
  \ControlFlowTok{for}\NormalTok{ (row }\ControlFlowTok{in} \DecValTok{1}\SpecialCharTok{:}\NormalTok{nGridPoints) \{}
    \ControlFlowTok{for}\NormalTok{ (col }\ControlFlowTok{in} \DecValTok{1}\SpecialCharTok{:}\NormalTok{nGridPoints) \{}
\NormalTok{      p1 }\OtherTok{=}\NormalTok{ pGrid[row]}
\NormalTok{      p2 }\OtherTok{=}\NormalTok{ pGrid[col]}
\NormalTok{      likelyhood }\OtherTok{=} \FunctionTok{dbinom}\NormalTok{(ball1\_success, n1, p1) }\SpecialCharTok{*} \FunctionTok{dbinom}\NormalTok{(ball2\_success, n2, p2)}
\NormalTok{      prior }\OtherTok{=}\NormalTok{ priorM[row, col]}
\NormalTok{      postM[row, col] }\OtherTok{=}\NormalTok{ likelyhood }\SpecialCharTok{*}\NormalTok{ prior}
\NormalTok{    \}}
\NormalTok{  \}}
\NormalTok{  postM }\OtherTok{=}\NormalTok{ postM }\SpecialCharTok{/}\NormalTok{ (}\FunctionTok{sum}\NormalTok{(postM) }\SpecialCharTok{*}\NormalTok{ gridSize }\SpecialCharTok{\^{}} \DecValTok{2}\NormalTok{)}
  \FunctionTok{return}\NormalTok{(postM)}
\NormalTok{\}}
\end{Highlighting}
\end{Shaded}

\hypertarget{step-1-fit-using-only-old-data}{%
\subsection{step 1: Fit using only old
data}\label{step-1-fit-using-only-old-data}}

\begin{Shaded}
\begin{Highlighting}[]
\NormalTok{ar\_data }\OtherTok{=}\NormalTok{ data[data}\SpecialCharTok{$}\NormalTok{tosser}\SpecialCharTok{==}\StringTok{"ar"}\NormalTok{, ]}
\NormalTok{ar\_ball1 }\OtherTok{=}\NormalTok{ ar\_data}\SpecialCharTok{$}\NormalTok{ball\_1}
\NormalTok{ar\_ball2 }\OtherTok{=}\NormalTok{ ar\_data}\SpecialCharTok{$}\NormalTok{ball\_2}
\NormalTok{arWater\_1 }\OtherTok{=} \FunctionTok{sum}\NormalTok{(ar\_ball1)}
\NormalTok{arWater\_2 }\OtherTok{=} \FunctionTok{sum}\NormalTok{(ar\_ball2)}
\NormalTok{ar\_n1 }\OtherTok{=} \FunctionTok{length}\NormalTok{(ar\_ball1)}
\NormalTok{ar\_n2 }\OtherTok{=} \FunctionTok{length}\NormalTok{(ar\_ball2)}

\CommentTok{\#Heat Map}
\NormalTok{priorM }\OtherTok{=} \FunctionTok{computeJointPrior}\NormalTok{()}
\NormalTok{postM }\OtherTok{=} \FunctionTok{computeJointPosterior}\NormalTok{(arWater\_1, arWater\_2, ar\_n1, ar\_n2, priorM)}
\FunctionTok{library}\NormalTok{(lattice)}
\NormalTok{new.palette}\OtherTok{=}\FunctionTok{colorRampPalette}\NormalTok{(}\FunctionTok{c}\NormalTok{(}\StringTok{"white"}\NormalTok{,}\StringTok{"red"}\NormalTok{,}\StringTok{"yellow"}\NormalTok{,}\StringTok{"white"}\NormalTok{),}\AttributeTok{space=}\StringTok{"rgb"}\NormalTok{)}
\FunctionTok{levelplot}\NormalTok{(}\AttributeTok{main=} \StringTok{"Joint Posterior Density"}\NormalTok{, }
\NormalTok{          postM, }\AttributeTok{col.regions=}\FunctionTok{new.palette}\NormalTok{(}\DecValTok{20}\NormalTok{),}
          \AttributeTok{xlab =} \StringTok{"p1"}\NormalTok{, }\AttributeTok{ylab =} \StringTok{"p2"}\NormalTok{,}
          \AttributeTok{scales=}\FunctionTok{list}\NormalTok{(}\AttributeTok{x=}\FunctionTok{list}\NormalTok{(}\AttributeTok{at=}\FunctionTok{c}\NormalTok{(}\DecValTok{50}\NormalTok{), }\AttributeTok{labels=}\FunctionTok{c}\NormalTok{(}\FloatTok{0.5}\NormalTok{)),}
                      \AttributeTok{y=}\FunctionTok{list}\NormalTok{(}\AttributeTok{at=}\FunctionTok{c}\NormalTok{(}\DecValTok{50}\NormalTok{), }\AttributeTok{labels=}\FunctionTok{c}\NormalTok{(}\FloatTok{0.5}\NormalTok{))),}
          \AttributeTok{panel =} \ControlFlowTok{function}\NormalTok{(...)\{}
            \FunctionTok{panel.levelplot}\NormalTok{(...)}
            \FunctionTok{panel.abline}\NormalTok{(}\DecValTok{0}\NormalTok{,}\DecValTok{1}\NormalTok{, }\AttributeTok{col =} \StringTok{"black"}\NormalTok{)}
            \FunctionTok{panel.abline}\NormalTok{(}\AttributeTok{v=}\DecValTok{50}\NormalTok{, }\AttributeTok{col =} \StringTok{"black"}\NormalTok{, }\AttributeTok{lty=}\DecValTok{2}\NormalTok{)}
            \FunctionTok{panel.abline}\NormalTok{(}\AttributeTok{h=}\DecValTok{50}\NormalTok{, }\AttributeTok{col =} \StringTok{"black"}\NormalTok{, }\AttributeTok{lty=}\DecValTok{2}\NormalTok{)\})}
\end{Highlighting}
\end{Shaded}

\includegraphics{ps2_template_files/figure-latex/unnamed-chunk-4-1.pdf}

\begin{Shaded}
\begin{Highlighting}[]
\CommentTok{\#Marginal Posterior}
\NormalTok{computePost }\OtherTok{\textless{}{-}} \ControlFlowTok{function}\NormalTok{(data, prior) \{}
\NormalTok{  success }\OtherTok{=} \FunctionTok{sum}\NormalTok{(data)}
\NormalTok{  trials }\OtherTok{=} \FunctionTok{length}\NormalTok{(data)}
\NormalTok{  likelyhood }\OtherTok{=} \FunctionTok{dbinom}\NormalTok{(}\AttributeTok{x =}\NormalTok{ success, }\AttributeTok{size =}\NormalTok{ trials, }\AttributeTok{prob =}\NormalTok{ pGrid)}
\NormalTok{  posterior }\OtherTok{=}\NormalTok{ likelyhood }\SpecialCharTok{*}\NormalTok{ prior}
  \FunctionTok{return}\NormalTok{(posterior }\SpecialCharTok{/}\NormalTok{ (}\FunctionTok{sum}\NormalTok{(posterior) }\SpecialCharTok{*}\NormalTok{ gridSize))}
\NormalTok{\}}
\NormalTok{prior }\OtherTok{=} \FunctionTok{dbeta}\NormalTok{(}\AttributeTok{x =}\NormalTok{ pGrid, }\AttributeTok{shape1 =}\NormalTok{ aPrior, }\AttributeTok{shape2 =}\NormalTok{ bPrior)}
\NormalTok{post\_ball1 }\OtherTok{=} \FunctionTok{computePost}\NormalTok{(ar\_ball1, prior)}
\NormalTok{post\_ball2 }\OtherTok{=} \FunctionTok{computePost}\NormalTok{(ar\_ball2, prior)}
\FunctionTok{plot}\NormalTok{(}\AttributeTok{main =} \StringTok{"Marginal Posterior Densities"}\NormalTok{, pGrid, post\_ball1, }
     \AttributeTok{type =} \StringTok{"l"}\NormalTok{, }\AttributeTok{lwd =} \DecValTok{3}\NormalTok{, }\AttributeTok{xlab =} \StringTok{"Theta"}\NormalTok{, }\AttributeTok{ylab =} \StringTok{"Posterior"}\NormalTok{)}
\FunctionTok{lines}\NormalTok{(pGrid, post\_ball2, }\AttributeTok{type =} \StringTok{"l"}\NormalTok{, }\AttributeTok{lwd =} \DecValTok{3}\NormalTok{, }\AttributeTok{col =} \StringTok{"red"}\NormalTok{)}
\FunctionTok{legend}\NormalTok{(}\StringTok{"topleft"}\NormalTok{, }\AttributeTok{legend =} \FunctionTok{c}\NormalTok{(}\StringTok{"p1"}\NormalTok{, }\StringTok{"p2"}\NormalTok{), }\AttributeTok{lty =} \DecValTok{1}\NormalTok{, }
       \AttributeTok{col =} \FunctionTok{c}\NormalTok{(}\StringTok{"black"}\NormalTok{, }\StringTok{"red"}\NormalTok{), }\AttributeTok{lwd =} \FunctionTok{c}\NormalTok{(}\DecValTok{5}\NormalTok{, }\DecValTok{5}\NormalTok{), }\AttributeTok{bty =} \StringTok{"n"}\NormalTok{)}
\end{Highlighting}
\end{Shaded}

\includegraphics{ps2_template_files/figure-latex/unnamed-chunk-5-1.pdf}

\begin{Shaded}
\begin{Highlighting}[]
\CommentTok{\#Mean and Standard Deviation and Posterior Probability that p1 \textless{} p2}
\NormalTok{posterior\_mean\_ball1 }\OtherTok{=} \FunctionTok{sum}\NormalTok{(pGrid }\SpecialCharTok{*}\NormalTok{ post\_ball1) }\SpecialCharTok{*}\NormalTok{ gridSize}
\NormalTok{posterior\_sd\_ball1 }\OtherTok{=} 
  \FunctionTok{sqrt}\NormalTok{(}\FunctionTok{sum}\NormalTok{((pGrid }\SpecialCharTok{{-}}\NormalTok{ posterior\_mean\_ball1)}\SpecialCharTok{\^{}}\DecValTok{2} \SpecialCharTok{*}\NormalTok{ post\_ball1) }\SpecialCharTok{*}\NormalTok{ gridSize)}
\NormalTok{posterior\_mean\_ball2 }\OtherTok{=} \FunctionTok{sum}\NormalTok{(pGrid }\SpecialCharTok{*}\NormalTok{ post\_ball2) }\SpecialCharTok{*}\NormalTok{ gridSize}
\NormalTok{posterior\_sd\_ball2 }\OtherTok{=}  
  \FunctionTok{sqrt}\NormalTok{(}\FunctionTok{sum}\NormalTok{((pGrid }\SpecialCharTok{{-}}\NormalTok{ posterior\_mean\_ball2)}\SpecialCharTok{\^{}}\DecValTok{2} \SpecialCharTok{*}\NormalTok{ post\_ball2) }\SpecialCharTok{*}\NormalTok{ gridSize)}
\FunctionTok{cat}\NormalTok{(}\StringTok{"}\SpecialCharTok{\textbackslash{}n}\StringTok{Mean for ball 1 posterior: "}\NormalTok{, }\FunctionTok{round}\NormalTok{(posterior\_mean\_ball1, }\DecValTok{5}\NormalTok{))}
\end{Highlighting}
\end{Shaded}

\begin{verbatim}
## 
## Mean for ball 1 posterior:  0.47273
\end{verbatim}

\begin{Shaded}
\begin{Highlighting}[]
\FunctionTok{cat}\NormalTok{(}\StringTok{"}\SpecialCharTok{\textbackslash{}n}\StringTok{SD for ball 1 posterior: "}\NormalTok{, }\FunctionTok{round}\NormalTok{(posterior\_sd\_ball1, }\DecValTok{5}\NormalTok{))}
\end{Highlighting}
\end{Shaded}

\begin{verbatim}
## 
## SD for ball 1 posterior:  0.04739
\end{verbatim}

\begin{Shaded}
\begin{Highlighting}[]
\FunctionTok{cat}\NormalTok{(}\StringTok{"}\SpecialCharTok{\textbackslash{}n}\StringTok{Mean for ball 2 posterior: "}\NormalTok{, }\FunctionTok{round}\NormalTok{(posterior\_mean\_ball2, }\DecValTok{5}\NormalTok{))}
\end{Highlighting}
\end{Shaded}

\begin{verbatim}
## 
## Mean for ball 2 posterior:  0.53636
\end{verbatim}

\begin{Shaded}
\begin{Highlighting}[]
\FunctionTok{cat}\NormalTok{(}\StringTok{"}\SpecialCharTok{\textbackslash{}n}\StringTok{SD for ball 1 posterior: "}\NormalTok{, }\FunctionTok{round}\NormalTok{(posterior\_sd\_ball2, }\DecValTok{5}\NormalTok{))}
\end{Highlighting}
\end{Shaded}

\begin{verbatim}
## 
## SD for ball 1 posterior:  0.04733
\end{verbatim}

\begin{Shaded}
\begin{Highlighting}[]
\CommentTok{\#Should we include diagonal???????}
\NormalTok{prob }\OtherTok{=} \FunctionTok{sum}\NormalTok{(postM[}\FunctionTok{upper.tri}\NormalTok{(postM)]) }\SpecialCharTok{*}\NormalTok{ (gridSize }\SpecialCharTok{\^{}} \DecValTok{2}\NormalTok{) }
\FunctionTok{cat}\NormalTok{(}\StringTok{"}\SpecialCharTok{\textbackslash{}n}\StringTok{Posterior Probability that p1 \textless{} p2: "}\NormalTok{, prob)}
\end{Highlighting}
\end{Shaded}

\begin{verbatim}
## 
## Posterior Probability that p1 < p2:  0.8089505
\end{verbatim}

\hypertarget{step-2-fit-only-using-new-data}{%
\subsection{step 2: Fit only using new
data}\label{step-2-fit-only-using-new-data}}

\begin{Shaded}
\begin{Highlighting}[]
\NormalTok{nh\_data }\OtherTok{=}\NormalTok{ data[data}\SpecialCharTok{$}\NormalTok{tosser}\SpecialCharTok{==}\StringTok{"nh"}\NormalTok{, ]}
\NormalTok{nh\_ball1 }\OtherTok{=}\NormalTok{ nh\_data}\SpecialCharTok{$}\NormalTok{ball\_1}
\NormalTok{nh\_ball2 }\OtherTok{=}\NormalTok{ nh\_data}\SpecialCharTok{$}\NormalTok{ball\_2}
\NormalTok{nhWater\_1 }\OtherTok{=} \FunctionTok{sum}\NormalTok{(nh\_ball1)}
\NormalTok{nhWater\_2 }\OtherTok{=} \FunctionTok{sum}\NormalTok{(nh\_ball2)}
\NormalTok{nh\_n1 }\OtherTok{=} \FunctionTok{length}\NormalTok{(nh\_ball1)}
\NormalTok{nh\_n2 }\OtherTok{=} \FunctionTok{length}\NormalTok{(nh\_ball2)}

\CommentTok{\#Heat Map}
\NormalTok{priorM }\OtherTok{=} \FunctionTok{computeJointPrior}\NormalTok{()}
\NormalTok{postM }\OtherTok{=} \FunctionTok{computeJointPosterior}\NormalTok{(nhWater\_1, nhWater\_2, nh\_n1, nh\_n2, priorM)}
\FunctionTok{library}\NormalTok{(lattice)}
\NormalTok{new.palette}\OtherTok{=}\FunctionTok{colorRampPalette}\NormalTok{(}\FunctionTok{c}\NormalTok{(}\StringTok{"white"}\NormalTok{,}\StringTok{"red"}\NormalTok{,}\StringTok{"yellow"}\NormalTok{,}\StringTok{"white"}\NormalTok{),}\AttributeTok{space=}\StringTok{"rgb"}\NormalTok{)}
\FunctionTok{levelplot}\NormalTok{(}\AttributeTok{main=} \StringTok{"Joint Posterior Density"}\NormalTok{, }
\NormalTok{          postM, }\AttributeTok{col.regions=}\FunctionTok{new.palette}\NormalTok{(}\DecValTok{20}\NormalTok{),}
          \AttributeTok{xlab =} \StringTok{"p1"}\NormalTok{, }\AttributeTok{ylab =} \StringTok{"p2"}\NormalTok{,}
          \AttributeTok{scales=}\FunctionTok{list}\NormalTok{(}\AttributeTok{x=}\FunctionTok{list}\NormalTok{(}\AttributeTok{at=}\FunctionTok{c}\NormalTok{(}\DecValTok{50}\NormalTok{), }\AttributeTok{labels=}\FunctionTok{c}\NormalTok{(}\FloatTok{0.5}\NormalTok{)),}
                      \AttributeTok{y=}\FunctionTok{list}\NormalTok{(}\AttributeTok{at=}\FunctionTok{c}\NormalTok{(}\DecValTok{50}\NormalTok{), }\AttributeTok{labels=}\FunctionTok{c}\NormalTok{(}\FloatTok{0.5}\NormalTok{))),}
          \AttributeTok{panel =} \ControlFlowTok{function}\NormalTok{(...)\{}
            \FunctionTok{panel.levelplot}\NormalTok{(...)}
            \FunctionTok{panel.abline}\NormalTok{(}\DecValTok{0}\NormalTok{,}\DecValTok{1}\NormalTok{, }\AttributeTok{col =} \StringTok{"black"}\NormalTok{)}
            \FunctionTok{panel.abline}\NormalTok{(}\AttributeTok{v=}\DecValTok{50}\NormalTok{, }\AttributeTok{col =} \StringTok{"black"}\NormalTok{, }\AttributeTok{lty=}\DecValTok{2}\NormalTok{)}
            \FunctionTok{panel.abline}\NormalTok{(}\AttributeTok{h=}\DecValTok{50}\NormalTok{, }\AttributeTok{col =} \StringTok{"black"}\NormalTok{, }\AttributeTok{lty=}\DecValTok{2}\NormalTok{)\})}
\end{Highlighting}
\end{Shaded}

\includegraphics{ps2_template_files/figure-latex/unnamed-chunk-7-1.pdf}

\begin{Shaded}
\begin{Highlighting}[]
\CommentTok{\#Marginal Posterior}
\NormalTok{prior }\OtherTok{=} \FunctionTok{dbeta}\NormalTok{(}\AttributeTok{x =}\NormalTok{ pGrid, }\AttributeTok{shape1 =}\NormalTok{ aPrior, }\AttributeTok{shape2 =}\NormalTok{ bPrior)}
\NormalTok{post\_ball1 }\OtherTok{=} \FunctionTok{computePost}\NormalTok{(nh\_ball1, prior)}
\NormalTok{post\_ball2 }\OtherTok{=} \FunctionTok{computePost}\NormalTok{(nh\_ball2, prior)}
\FunctionTok{plot}\NormalTok{(}\AttributeTok{main =} \StringTok{"Marginal Posterior Densities"}\NormalTok{, pGrid, post\_ball1, }
     \AttributeTok{type =} \StringTok{"l"}\NormalTok{, }\AttributeTok{lwd =} \DecValTok{3}\NormalTok{, }\AttributeTok{xlab =} \StringTok{"Theta"}\NormalTok{, }\AttributeTok{ylab =} \StringTok{"Posterior"}\NormalTok{)}
\FunctionTok{lines}\NormalTok{(pGrid, post\_ball2, }\AttributeTok{type =} \StringTok{"l"}\NormalTok{, }\AttributeTok{lwd =} \DecValTok{3}\NormalTok{, }\AttributeTok{col =} \StringTok{"red"}\NormalTok{)}
\FunctionTok{legend}\NormalTok{(}\StringTok{"topleft"}\NormalTok{, }\AttributeTok{legend =} \FunctionTok{c}\NormalTok{(}\StringTok{"p1"}\NormalTok{, }\StringTok{"p2"}\NormalTok{), }\AttributeTok{lty =} \DecValTok{1}\NormalTok{, }
       \AttributeTok{col =} \FunctionTok{c}\NormalTok{(}\StringTok{"black"}\NormalTok{, }\StringTok{"red"}\NormalTok{), }\AttributeTok{lwd =} \FunctionTok{c}\NormalTok{(}\DecValTok{5}\NormalTok{, }\DecValTok{5}\NormalTok{), }\AttributeTok{bty =} \StringTok{"n"}\NormalTok{)}
\end{Highlighting}
\end{Shaded}

\includegraphics{ps2_template_files/figure-latex/unnamed-chunk-8-1.pdf}

\begin{Shaded}
\begin{Highlighting}[]
\CommentTok{\#Mean and Standard Deviation and Posterior Probability that p1 \textless{} p2}
\NormalTok{posterior\_mean\_ball1 }\OtherTok{=} \FunctionTok{sum}\NormalTok{(pGrid }\SpecialCharTok{*}\NormalTok{ post\_ball1) }\SpecialCharTok{*}\NormalTok{ gridSize}
\NormalTok{posterior\_sd\_ball1 }\OtherTok{=} 
  \FunctionTok{sqrt}\NormalTok{(}\FunctionTok{sum}\NormalTok{((pGrid }\SpecialCharTok{{-}}\NormalTok{ posterior\_mean\_ball1)}\SpecialCharTok{\^{}}\DecValTok{2} \SpecialCharTok{*}\NormalTok{ post\_ball1) }\SpecialCharTok{*}\NormalTok{ gridSize)}
\NormalTok{posterior\_mean\_ball2 }\OtherTok{=} \FunctionTok{sum}\NormalTok{(pGrid }\SpecialCharTok{*}\NormalTok{ post\_ball2) }\SpecialCharTok{*}\NormalTok{ gridSize}
\NormalTok{posterior\_sd\_ball2 }\OtherTok{=}  
  \FunctionTok{sqrt}\NormalTok{(}\FunctionTok{sum}\NormalTok{((pGrid }\SpecialCharTok{{-}}\NormalTok{ posterior\_mean\_ball2)}\SpecialCharTok{\^{}}\DecValTok{2} \SpecialCharTok{*}\NormalTok{ post\_ball2) }\SpecialCharTok{*}\NormalTok{ gridSize)}
\FunctionTok{cat}\NormalTok{(}\StringTok{"}\SpecialCharTok{\textbackslash{}n}\StringTok{Mean for ball 1 posterior: "}\NormalTok{, }\FunctionTok{round}\NormalTok{(posterior\_mean\_ball1, }\DecValTok{5}\NormalTok{))}
\end{Highlighting}
\end{Shaded}

\begin{verbatim}
## 
## Mean for ball 1 posterior:  0.43636
\end{verbatim}

\begin{Shaded}
\begin{Highlighting}[]
\FunctionTok{cat}\NormalTok{(}\StringTok{"}\SpecialCharTok{\textbackslash{}n}\StringTok{SD for ball 1 posterior: "}\NormalTok{, }\FunctionTok{round}\NormalTok{(posterior\_sd\_ball1, }\DecValTok{5}\NormalTok{))}
\end{Highlighting}
\end{Shaded}

\begin{verbatim}
## 
## SD for ball 1 posterior:  0.04707
\end{verbatim}

\begin{Shaded}
\begin{Highlighting}[]
\FunctionTok{cat}\NormalTok{(}\StringTok{"}\SpecialCharTok{\textbackslash{}n}\StringTok{Mean for ball 2 posterior: "}\NormalTok{, }\FunctionTok{round}\NormalTok{(posterior\_mean\_ball2, }\DecValTok{5}\NormalTok{))}
\end{Highlighting}
\end{Shaded}

\begin{verbatim}
## 
## Mean for ball 2 posterior:  0.44545
\end{verbatim}

\begin{Shaded}
\begin{Highlighting}[]
\FunctionTok{cat}\NormalTok{(}\StringTok{"}\SpecialCharTok{\textbackslash{}n}\StringTok{SD for ball 1 posterior: "}\NormalTok{, }\FunctionTok{round}\NormalTok{(posterior\_sd\_ball2, }\DecValTok{5}\NormalTok{))}
\end{Highlighting}
\end{Shaded}

\begin{verbatim}
## 
## SD for ball 1 posterior:  0.04717
\end{verbatim}

\begin{Shaded}
\begin{Highlighting}[]
\CommentTok{\#Should we include diagonal???????}
\NormalTok{prob }\OtherTok{=} \FunctionTok{sum}\NormalTok{(postM[}\FunctionTok{upper.tri}\NormalTok{(postM)]) }\SpecialCharTok{*}\NormalTok{ (gridSize }\SpecialCharTok{\^{}} \DecValTok{2}\NormalTok{) }
\FunctionTok{cat}\NormalTok{(}\StringTok{"}\SpecialCharTok{\textbackslash{}n}\StringTok{Posterior Probability that p1 \textless{} p2: "}\NormalTok{, prob)}
\end{Highlighting}
\end{Shaded}

\begin{verbatim}
## 
## Posterior Probability that p1 < p2:  0.5242769
\end{verbatim}

\hypertarget{step-3-fit-using-all-the-data}{%
\subsection{Step 3: Fit using all the
data}\label{step-3-fit-using-all-the-data}}

\begin{Shaded}
\begin{Highlighting}[]
\NormalTok{ball1 }\OtherTok{=}\NormalTok{ data}\SpecialCharTok{$}\NormalTok{ball\_1}
\NormalTok{ball2 }\OtherTok{=}\NormalTok{ data}\SpecialCharTok{$}\NormalTok{ball\_2}
\NormalTok{success\_ball1 }\OtherTok{=} \FunctionTok{sum}\NormalTok{(ball1)}
\NormalTok{success\_ball2 }\OtherTok{=} \FunctionTok{sum}\NormalTok{(ball2)}
\NormalTok{n\_ball1 }\OtherTok{=} \FunctionTok{length}\NormalTok{(ball1)}
\NormalTok{n\_ball2 }\OtherTok{=} \FunctionTok{length}\NormalTok{(ball2)}

\CommentTok{\#Heat Map}
\NormalTok{priorM }\OtherTok{=} \FunctionTok{computeJointPrior}\NormalTok{()}
\NormalTok{postM }\OtherTok{=} \FunctionTok{computeJointPosterior}\NormalTok{(success\_ball1, success\_ball2, n\_ball1, n\_ball2, priorM)}
\FunctionTok{library}\NormalTok{(lattice)}
\NormalTok{new.palette}\OtherTok{=}\FunctionTok{colorRampPalette}\NormalTok{(}\FunctionTok{c}\NormalTok{(}\StringTok{"white"}\NormalTok{,}\StringTok{"red"}\NormalTok{,}\StringTok{"yellow"}\NormalTok{,}\StringTok{"white"}\NormalTok{),}\AttributeTok{space=}\StringTok{"rgb"}\NormalTok{)}
\FunctionTok{levelplot}\NormalTok{(}\AttributeTok{main=} \StringTok{"Joint Posterior Density"}\NormalTok{, }
\NormalTok{          postM, }\AttributeTok{col.regions=}\FunctionTok{new.palette}\NormalTok{(}\DecValTok{20}\NormalTok{),}
          \AttributeTok{xlab =} \StringTok{"p1"}\NormalTok{, }\AttributeTok{ylab =} \StringTok{"p2"}\NormalTok{,}
          \AttributeTok{scales=}\FunctionTok{list}\NormalTok{(}\AttributeTok{x=}\FunctionTok{list}\NormalTok{(}\AttributeTok{at=}\FunctionTok{c}\NormalTok{(}\DecValTok{50}\NormalTok{), }\AttributeTok{labels=}\FunctionTok{c}\NormalTok{(}\FloatTok{0.5}\NormalTok{)),}
                      \AttributeTok{y=}\FunctionTok{list}\NormalTok{(}\AttributeTok{at=}\FunctionTok{c}\NormalTok{(}\DecValTok{50}\NormalTok{), }\AttributeTok{labels=}\FunctionTok{c}\NormalTok{(}\FloatTok{0.5}\NormalTok{))),}
          \AttributeTok{panel =} \ControlFlowTok{function}\NormalTok{(...)\{}
            \FunctionTok{panel.levelplot}\NormalTok{(...)}
            \FunctionTok{panel.abline}\NormalTok{(}\DecValTok{0}\NormalTok{,}\DecValTok{1}\NormalTok{, }\AttributeTok{col =} \StringTok{"black"}\NormalTok{)}
            \FunctionTok{panel.abline}\NormalTok{(}\AttributeTok{v=}\DecValTok{50}\NormalTok{, }\AttributeTok{col =} \StringTok{"black"}\NormalTok{, }\AttributeTok{lty=}\DecValTok{2}\NormalTok{)}
            \FunctionTok{panel.abline}\NormalTok{(}\AttributeTok{h=}\DecValTok{50}\NormalTok{, }\AttributeTok{col =} \StringTok{"black"}\NormalTok{, }\AttributeTok{lty=}\DecValTok{2}\NormalTok{)\})}
\end{Highlighting}
\end{Shaded}

\includegraphics{ps2_template_files/figure-latex/unnamed-chunk-10-1.pdf}

\begin{Shaded}
\begin{Highlighting}[]
\NormalTok{prior }\OtherTok{=} \FunctionTok{dbeta}\NormalTok{(}\AttributeTok{x =}\NormalTok{ pGrid, }\AttributeTok{shape1 =}\NormalTok{ aPrior, }\AttributeTok{shape2 =}\NormalTok{ bPrior)}
\NormalTok{post\_ball1 }\OtherTok{=} \FunctionTok{computePost}\NormalTok{(ball1, prior)}
\NormalTok{post\_ball2 }\OtherTok{=} \FunctionTok{computePost}\NormalTok{(ball2, prior)}
\FunctionTok{plot}\NormalTok{(}\AttributeTok{main =} \StringTok{"Marginal Posterior Densities"}\NormalTok{, pGrid, post\_ball1, }
     \AttributeTok{type =} \StringTok{"l"}\NormalTok{, }\AttributeTok{lwd =} \DecValTok{3}\NormalTok{, }\AttributeTok{xlab =} \StringTok{"Theta"}\NormalTok{, }\AttributeTok{ylab =} \StringTok{"Posterior"}\NormalTok{)}
\FunctionTok{lines}\NormalTok{(pGrid, post\_ball2, }\AttributeTok{type =} \StringTok{"l"}\NormalTok{, }\AttributeTok{lwd =} \DecValTok{3}\NormalTok{, }\AttributeTok{col =} \StringTok{"red"}\NormalTok{)}
\FunctionTok{legend}\NormalTok{(}\StringTok{"topleft"}\NormalTok{, }\AttributeTok{legend =} \FunctionTok{c}\NormalTok{(}\StringTok{"p1"}\NormalTok{, }\StringTok{"p2"}\NormalTok{), }\AttributeTok{lty =} \DecValTok{1}\NormalTok{, }
       \AttributeTok{col =} \FunctionTok{c}\NormalTok{(}\StringTok{"black"}\NormalTok{, }\StringTok{"red"}\NormalTok{), }\AttributeTok{lwd =} \FunctionTok{c}\NormalTok{(}\DecValTok{5}\NormalTok{, }\DecValTok{5}\NormalTok{), }\AttributeTok{bty =} \StringTok{"n"}\NormalTok{)}
\end{Highlighting}
\end{Shaded}

\includegraphics{ps2_template_files/figure-latex/unnamed-chunk-11-1.pdf}

\begin{Shaded}
\begin{Highlighting}[]
\CommentTok{\#Mean and Standard Deviation and Posterior Probability that p1 \textless{} p2}
\NormalTok{posterior\_mean\_ball1 }\OtherTok{=} \FunctionTok{sum}\NormalTok{(pGrid }\SpecialCharTok{*}\NormalTok{ post\_ball1) }\SpecialCharTok{*}\NormalTok{ gridSize}
\NormalTok{posterior\_sd\_ball1 }\OtherTok{=} 
  \FunctionTok{sqrt}\NormalTok{(}\FunctionTok{sum}\NormalTok{((pGrid }\SpecialCharTok{{-}}\NormalTok{ posterior\_mean\_ball1)}\SpecialCharTok{\^{}}\DecValTok{2} \SpecialCharTok{*}\NormalTok{ post\_ball1) }\SpecialCharTok{*}\NormalTok{ gridSize)}
\NormalTok{posterior\_mean\_ball2 }\OtherTok{=} \FunctionTok{sum}\NormalTok{(pGrid }\SpecialCharTok{*}\NormalTok{ post\_ball2) }\SpecialCharTok{*}\NormalTok{ gridSize}
\NormalTok{posterior\_sd\_ball2 }\OtherTok{=}  
  \FunctionTok{sqrt}\NormalTok{(}\FunctionTok{sum}\NormalTok{((pGrid }\SpecialCharTok{{-}}\NormalTok{ posterior\_mean\_ball2)}\SpecialCharTok{\^{}}\DecValTok{2} \SpecialCharTok{*}\NormalTok{ post\_ball2) }\SpecialCharTok{*}\NormalTok{ gridSize)}
\FunctionTok{cat}\NormalTok{(}\StringTok{"}\SpecialCharTok{\textbackslash{}n}\StringTok{Mean for ball 1 posterior: "}\NormalTok{, }\FunctionTok{round}\NormalTok{(posterior\_mean\_ball1, }\DecValTok{5}\NormalTok{))}
\end{Highlighting}
\end{Shaded}

\begin{verbatim}
## 
## Mean for ball 1 posterior:  0.45238
\end{verbatim}

\begin{Shaded}
\begin{Highlighting}[]
\FunctionTok{cat}\NormalTok{(}\StringTok{"}\SpecialCharTok{\textbackslash{}n}\StringTok{SD for ball 1 posterior: "}\NormalTok{, }\FunctionTok{round}\NormalTok{(posterior\_sd\_ball1, }\DecValTok{5}\NormalTok{))}
\end{Highlighting}
\end{Shaded}

\begin{verbatim}
## 
## SD for ball 1 posterior:  0.03426
\end{verbatim}

\begin{Shaded}
\begin{Highlighting}[]
\FunctionTok{cat}\NormalTok{(}\StringTok{"}\SpecialCharTok{\textbackslash{}n}\StringTok{Mean for ball 2 posterior: "}\NormalTok{, }\FunctionTok{round}\NormalTok{(posterior\_mean\_ball2, }\DecValTok{5}\NormalTok{))}
\end{Highlighting}
\end{Shaded}

\begin{verbatim}
## 
## Mean for ball 2 posterior:  0.49048
\end{verbatim}

\begin{Shaded}
\begin{Highlighting}[]
\FunctionTok{cat}\NormalTok{(}\StringTok{"}\SpecialCharTok{\textbackslash{}n}\StringTok{SD for ball 1 posterior: "}\NormalTok{, }\FunctionTok{round}\NormalTok{(posterior\_sd\_ball2, }\DecValTok{5}\NormalTok{))}
\end{Highlighting}
\end{Shaded}

\begin{verbatim}
## 
## SD for ball 1 posterior:  0.03442
\end{verbatim}

\begin{Shaded}
\begin{Highlighting}[]
\CommentTok{\#Should we include diagonal???????}
\NormalTok{prob }\OtherTok{=} \FunctionTok{sum}\NormalTok{(postM[}\FunctionTok{upper.tri}\NormalTok{(postM)]) }\SpecialCharTok{*}\NormalTok{ (gridSize }\SpecialCharTok{\^{}} \DecValTok{2}\NormalTok{) }
\FunctionTok{cat}\NormalTok{(}\StringTok{"}\SpecialCharTok{\textbackslash{}n}\StringTok{Posterior Probability that p1 \textless{} p2: "}\NormalTok{, prob)}
\end{Highlighting}
\end{Shaded}

\begin{verbatim}
## 
## Posterior Probability that p1 < p2:  0.7521635
\end{verbatim}

\hypertarget{part-2}{%
\section{PART 2}\label{part-2}}

\hypertarget{step-1}{%
\subsection{step 1}\label{step-1}}

\hypertarget{a}{%
\subsubsection{A}\label{a}}

\begin{Shaded}
\begin{Highlighting}[]
\CommentTok{\# Enter your code here.}
\end{Highlighting}
\end{Shaded}

\hypertarget{b}{%
\subsubsection{B}\label{b}}

\begin{Shaded}
\begin{Highlighting}[]
\CommentTok{\# Enter your code here.}
\end{Highlighting}
\end{Shaded}

\hypertarget{c}{%
\subsubsection{C}\label{c}}

\begin{Shaded}
\begin{Highlighting}[]
\CommentTok{\# Enter your code here.}
\end{Highlighting}
\end{Shaded}

\hypertarget{step-2}{%
\subsection{Step 2}\label{step-2}}

\hypertarget{a-1}{%
\subsection{A}\label{a-1}}

\begin{Shaded}
\begin{Highlighting}[]
\CommentTok{\# Enter your code here.}
\end{Highlighting}
\end{Shaded}

\hypertarget{b-1}{%
\subsubsection{B}\label{b-1}}

\begin{Shaded}
\begin{Highlighting}[]
\CommentTok{\# Enter your code here.}
\end{Highlighting}
\end{Shaded}

\hypertarget{c-1}{%
\subsubsection{C}\label{c-1}}

\begin{Shaded}
\begin{Highlighting}[]
\CommentTok{\# Enter your code here.}
\end{Highlighting}
\end{Shaded}

\hypertarget{step-3}{%
\subsection{Step 3}\label{step-3}}

Discuss solution here.

\hypertarget{part-3}{%
\section{PART 3}\label{part-3}}

\hypertarget{step-1-1}{%
\subsection{Step 1}\label{step-1-1}}

\begin{Shaded}
\begin{Highlighting}[]
\CommentTok{\# Enter your code here.}
\end{Highlighting}
\end{Shaded}

\hypertarget{step-2-1}{%
\subsection{Step 2}\label{step-2-1}}

Discuss here.

\end{document}

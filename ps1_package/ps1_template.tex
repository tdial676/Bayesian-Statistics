% Options for packages loaded elsewhere
\PassOptionsToPackage{unicode}{hyperref}
\PassOptionsToPackage{hyphens}{url}
%
\documentclass[
]{article}
\usepackage{amsmath,amssymb}
\usepackage{iftex}
\ifPDFTeX
  \usepackage[T1]{fontenc}
  \usepackage[utf8]{inputenc}
  \usepackage{textcomp} % provide euro and other symbols
\else % if luatex or xetex
  \usepackage{unicode-math} % this also loads fontspec
  \defaultfontfeatures{Scale=MatchLowercase}
  \defaultfontfeatures[\rmfamily]{Ligatures=TeX,Scale=1}
\fi
\usepackage{lmodern}
\ifPDFTeX\else
  % xetex/luatex font selection
\fi
% Use upquote if available, for straight quotes in verbatim environments
\IfFileExists{upquote.sty}{\usepackage{upquote}}{}
\IfFileExists{microtype.sty}{% use microtype if available
  \usepackage[]{microtype}
  \UseMicrotypeSet[protrusion]{basicmath} % disable protrusion for tt fonts
}{}
\makeatletter
\@ifundefined{KOMAClassName}{% if non-KOMA class
  \IfFileExists{parskip.sty}{%
    \usepackage{parskip}
  }{% else
    \setlength{\parindent}{0pt}
    \setlength{\parskip}{6pt plus 2pt minus 1pt}}
}{% if KOMA class
  \KOMAoptions{parskip=half}}
\makeatother
\usepackage{xcolor}
\usepackage[margin=1in]{geometry}
\usepackage{color}
\usepackage{fancyvrb}
\newcommand{\VerbBar}{|}
\newcommand{\VERB}{\Verb[commandchars=\\\{\}]}
\DefineVerbatimEnvironment{Highlighting}{Verbatim}{commandchars=\\\{\}}
% Add ',fontsize=\small' for more characters per line
\usepackage{framed}
\definecolor{shadecolor}{RGB}{248,248,248}
\newenvironment{Shaded}{\begin{snugshade}}{\end{snugshade}}
\newcommand{\AlertTok}[1]{\textcolor[rgb]{0.94,0.16,0.16}{#1}}
\newcommand{\AnnotationTok}[1]{\textcolor[rgb]{0.56,0.35,0.01}{\textbf{\textit{#1}}}}
\newcommand{\AttributeTok}[1]{\textcolor[rgb]{0.13,0.29,0.53}{#1}}
\newcommand{\BaseNTok}[1]{\textcolor[rgb]{0.00,0.00,0.81}{#1}}
\newcommand{\BuiltInTok}[1]{#1}
\newcommand{\CharTok}[1]{\textcolor[rgb]{0.31,0.60,0.02}{#1}}
\newcommand{\CommentTok}[1]{\textcolor[rgb]{0.56,0.35,0.01}{\textit{#1}}}
\newcommand{\CommentVarTok}[1]{\textcolor[rgb]{0.56,0.35,0.01}{\textbf{\textit{#1}}}}
\newcommand{\ConstantTok}[1]{\textcolor[rgb]{0.56,0.35,0.01}{#1}}
\newcommand{\ControlFlowTok}[1]{\textcolor[rgb]{0.13,0.29,0.53}{\textbf{#1}}}
\newcommand{\DataTypeTok}[1]{\textcolor[rgb]{0.13,0.29,0.53}{#1}}
\newcommand{\DecValTok}[1]{\textcolor[rgb]{0.00,0.00,0.81}{#1}}
\newcommand{\DocumentationTok}[1]{\textcolor[rgb]{0.56,0.35,0.01}{\textbf{\textit{#1}}}}
\newcommand{\ErrorTok}[1]{\textcolor[rgb]{0.64,0.00,0.00}{\textbf{#1}}}
\newcommand{\ExtensionTok}[1]{#1}
\newcommand{\FloatTok}[1]{\textcolor[rgb]{0.00,0.00,0.81}{#1}}
\newcommand{\FunctionTok}[1]{\textcolor[rgb]{0.13,0.29,0.53}{\textbf{#1}}}
\newcommand{\ImportTok}[1]{#1}
\newcommand{\InformationTok}[1]{\textcolor[rgb]{0.56,0.35,0.01}{\textbf{\textit{#1}}}}
\newcommand{\KeywordTok}[1]{\textcolor[rgb]{0.13,0.29,0.53}{\textbf{#1}}}
\newcommand{\NormalTok}[1]{#1}
\newcommand{\OperatorTok}[1]{\textcolor[rgb]{0.81,0.36,0.00}{\textbf{#1}}}
\newcommand{\OtherTok}[1]{\textcolor[rgb]{0.56,0.35,0.01}{#1}}
\newcommand{\PreprocessorTok}[1]{\textcolor[rgb]{0.56,0.35,0.01}{\textit{#1}}}
\newcommand{\RegionMarkerTok}[1]{#1}
\newcommand{\SpecialCharTok}[1]{\textcolor[rgb]{0.81,0.36,0.00}{\textbf{#1}}}
\newcommand{\SpecialStringTok}[1]{\textcolor[rgb]{0.31,0.60,0.02}{#1}}
\newcommand{\StringTok}[1]{\textcolor[rgb]{0.31,0.60,0.02}{#1}}
\newcommand{\VariableTok}[1]{\textcolor[rgb]{0.00,0.00,0.00}{#1}}
\newcommand{\VerbatimStringTok}[1]{\textcolor[rgb]{0.31,0.60,0.02}{#1}}
\newcommand{\WarningTok}[1]{\textcolor[rgb]{0.56,0.35,0.01}{\textbf{\textit{#1}}}}
\usepackage{graphicx}
\makeatletter
\def\maxwidth{\ifdim\Gin@nat@width>\linewidth\linewidth\else\Gin@nat@width\fi}
\def\maxheight{\ifdim\Gin@nat@height>\textheight\textheight\else\Gin@nat@height\fi}
\makeatother
% Scale images if necessary, so that they will not overflow the page
% margins by default, and it is still possible to overwrite the defaults
% using explicit options in \includegraphics[width, height, ...]{}
\setkeys{Gin}{width=\maxwidth,height=\maxheight,keepaspectratio}
% Set default figure placement to htbp
\makeatletter
\def\fps@figure{htbp}
\makeatother
\setlength{\emergencystretch}{3em} % prevent overfull lines
\providecommand{\tightlist}{%
  \setlength{\itemsep}{0pt}\setlength{\parskip}{0pt}}
\setcounter{secnumdepth}{-\maxdimen} % remove section numbering
\ifLuaTeX
  \usepackage{selnolig}  % disable illegal ligatures
\fi
\IfFileExists{bookmark.sty}{\usepackage{bookmark}}{\usepackage{hyperref}}
\IfFileExists{xurl.sty}{\usepackage{xurl}}{} % add URL line breaks if available
\urlstyle{same}
\hypersetup{
  pdftitle={Ec/ACM/CS 112. Problem set 1. Template},
  hidelinks,
  pdfcreator={LaTeX via pandoc}}

\title{Ec/ACM/CS 112. Problem set 1. Template}
\author{}
\date{\vspace{-2.5em}}

\begin{document}
\maketitle

\hypertarget{step-1-programming-stochastic-simulations}{%
\section{Step 1: Programming stochastic
simulations}\label{step-1-programming-stochastic-simulations}}

\hypertarget{a}{%
\subsection{1.A}\label{a}}

\begin{Shaded}
\begin{Highlighting}[]
\NormalTok{x1 }\OtherTok{=} \FunctionTok{rnorm}\NormalTok{(}\DecValTok{1000}\NormalTok{, }\AttributeTok{mean =} \DecValTok{10}\NormalTok{, }\AttributeTok{sd =} \DecValTok{10}\NormalTok{)}
\NormalTok{x2 }\OtherTok{=} \FunctionTok{rnorm}\NormalTok{(}\DecValTok{1000}\NormalTok{, }\AttributeTok{mean =} \DecValTok{10}\NormalTok{, }\AttributeTok{sd =} \DecValTok{10}\NormalTok{)}
\NormalTok{mean\_x1 }\OtherTok{=} \FunctionTok{mean}\NormalTok{(x1)}
\NormalTok{mean\_x2 }\OtherTok{=} \FunctionTok{mean}\NormalTok{(x2)}
\FunctionTok{plot}\NormalTok{(x1, x2, }\AttributeTok{main =} \StringTok{"random seed unspecified"}\NormalTok{, }\AttributeTok{xlim=}\FunctionTok{c}\NormalTok{(}\SpecialCharTok{{-}}\DecValTok{50}\NormalTok{, }\DecValTok{50}\NormalTok{), }
     \AttributeTok{ylim=}\FunctionTok{c}\NormalTok{(}\SpecialCharTok{{-}}\DecValTok{50}\NormalTok{, }\DecValTok{50}\NormalTok{), }\AttributeTok{pch=}\DecValTok{16}\NormalTok{, }\AttributeTok{col=}\FunctionTok{rgb}\NormalTok{(}\DecValTok{1}\NormalTok{, }\DecValTok{0}\NormalTok{, }\DecValTok{0}\NormalTok{, }\AttributeTok{alpha =} \FloatTok{0.3}\NormalTok{))}
\FunctionTok{abline}\NormalTok{(}\AttributeTok{h =}\NormalTok{ mean\_x2, }\AttributeTok{lty=}\DecValTok{2}\NormalTok{)}
\FunctionTok{abline}\NormalTok{(}\AttributeTok{v =}\NormalTok{ mean\_x1, }\AttributeTok{lty=}\DecValTok{2}\NormalTok{)}
\FunctionTok{text}\NormalTok{(}\SpecialCharTok{{-}}\DecValTok{35}\NormalTok{, }\SpecialCharTok{{-}}\DecValTok{5}\NormalTok{, }\FunctionTok{paste}\NormalTok{(}\StringTok{"mean x1 = "}\NormalTok{, }\FunctionTok{round}\NormalTok{(mean\_x1, }\DecValTok{2}\NormalTok{)))}
\FunctionTok{text}\NormalTok{(}\SpecialCharTok{{-}}\DecValTok{33}\NormalTok{, }\SpecialCharTok{{-}}\DecValTok{10}\NormalTok{, }\FunctionTok{paste}\NormalTok{(}\StringTok{"mean x2 = "}\NormalTok{, }\FunctionTok{round}\NormalTok{(mean\_x2, }\DecValTok{2}\NormalTok{)))}
\end{Highlighting}
\end{Shaded}

\includegraphics{ps1_template_files/figure-latex/unnamed-chunk-1-1.pdf}
\#\# 1.B

\begin{Shaded}
\begin{Highlighting}[]
\FunctionTok{set.seed}\NormalTok{(}\DecValTok{2021}\NormalTok{)}
\NormalTok{x1 }\OtherTok{=} \FunctionTok{rnorm}\NormalTok{(}\DecValTok{1000}\NormalTok{, }\AttributeTok{mean =} \DecValTok{10}\NormalTok{, }\AttributeTok{sd =} \DecValTok{10}\NormalTok{)}
\FunctionTok{set.seed}\NormalTok{(}\DecValTok{2021}\NormalTok{)}
\NormalTok{x2 }\OtherTok{=} \FunctionTok{rnorm}\NormalTok{(}\DecValTok{1000}\NormalTok{, }\AttributeTok{mean =} \DecValTok{10}\NormalTok{, }\AttributeTok{sd =} \DecValTok{10}\NormalTok{)}
\NormalTok{mean\_x1 }\OtherTok{=} \FunctionTok{mean}\NormalTok{(x1)}
\NormalTok{mean\_x2 }\OtherTok{=} \FunctionTok{mean}\NormalTok{(x2)}
\FunctionTok{plot}\NormalTok{(x1, x2, }\AttributeTok{main =} \StringTok{"random seed specified"}\NormalTok{, }\AttributeTok{xlim=}\FunctionTok{c}\NormalTok{(}\SpecialCharTok{{-}}\DecValTok{50}\NormalTok{, }\DecValTok{50}\NormalTok{), }
     \AttributeTok{ylim=}\FunctionTok{c}\NormalTok{(}\SpecialCharTok{{-}}\DecValTok{50}\NormalTok{, }\DecValTok{50}\NormalTok{), }\AttributeTok{pch=}\DecValTok{16}\NormalTok{, }\AttributeTok{col=}\FunctionTok{rgb}\NormalTok{(}\DecValTok{1}\NormalTok{, }\DecValTok{0}\NormalTok{, }\DecValTok{0}\NormalTok{, }\AttributeTok{alpha =} \FloatTok{0.3}\NormalTok{))}
\FunctionTok{abline}\NormalTok{(}\AttributeTok{a=}\DecValTok{0}\NormalTok{, }\AttributeTok{b=}\DecValTok{1}\NormalTok{, }\AttributeTok{lty=}\DecValTok{2}\NormalTok{)}
\FunctionTok{text}\NormalTok{(}\SpecialCharTok{{-}}\DecValTok{35}\NormalTok{, }\SpecialCharTok{{-}}\DecValTok{5}\NormalTok{, }\FunctionTok{paste}\NormalTok{(}\StringTok{"mean x1 = "}\NormalTok{, }\FunctionTok{round}\NormalTok{(mean\_x1, }\DecValTok{2}\NormalTok{)))}
\FunctionTok{text}\NormalTok{(}\SpecialCharTok{{-}}\DecValTok{33}\NormalTok{, }\SpecialCharTok{{-}}\DecValTok{10}\NormalTok{, }\FunctionTok{paste}\NormalTok{(}\StringTok{"mean x2 = "}\NormalTok{, }\FunctionTok{round}\NormalTok{(mean\_x2, }\DecValTok{2}\NormalTok{)))}
\end{Highlighting}
\end{Shaded}

\includegraphics{ps1_template_files/figure-latex/unnamed-chunk-2-1.pdf}
\# Step 2: Simulating the hot-hand in basketball

\hypertarget{a-1}{%
\subsection{2.A}\label{a-1}}

\begin{Shaded}
\begin{Highlighting}[]
\NormalTok{simulate\_player }\OtherTok{\textless{}{-}} \ControlFlowTok{function}\NormalTok{(p\_score) \{}
  \FunctionTok{return}\NormalTok{ (}\FunctionTok{rbinom}\NormalTok{(}\DecValTok{25}\NormalTok{, }\DecValTok{1}\NormalTok{, p\_score))}
\NormalTok{\}}

\NormalTok{count\_sequence }\OtherTok{\textless{}{-}} \ControlFlowTok{function}\NormalTok{(shots) \{}
\NormalTok{  max\_streak }\OtherTok{=} \DecValTok{0}
\NormalTok{  curr\_streak }\OtherTok{=} \DecValTok{0}
  \ControlFlowTok{for}\NormalTok{ (shot }\ControlFlowTok{in}\NormalTok{ shots) \{}
    \ControlFlowTok{if}\NormalTok{ (shot }\SpecialCharTok{==} \DecValTok{0}\NormalTok{) \{}
\NormalTok{      max\_streak }\OtherTok{=} \FunctionTok{max}\NormalTok{(curr\_streak, max\_streak)}
\NormalTok{      curr\_streak }\OtherTok{=} \DecValTok{0}
\NormalTok{    \}}
    \ControlFlowTok{else}\NormalTok{ \{}
\NormalTok{      curr\_streak }\OtherTok{=}\NormalTok{ curr\_streak }\SpecialCharTok{+} \DecValTok{1}
\NormalTok{    \}}
\NormalTok{  \}}
  \FunctionTok{return}\NormalTok{ (max\_streak)}
\NormalTok{\}}
\end{Highlighting}
\end{Shaded}

\hypertarget{b}{%
\subsection{2.B}\label{b}}

\begin{Shaded}
\begin{Highlighting}[]
\FunctionTok{set.seed}\NormalTok{(}\DecValTok{2024}\NormalTok{)}
\NormalTok{x }\OtherTok{=} \FunctionTok{c}\NormalTok{()}
\ControlFlowTok{for}\NormalTok{ (i }\ControlFlowTok{in} \DecValTok{1}\SpecialCharTok{:}\DecValTok{10000}\NormalTok{) \{}
\NormalTok{  x[i] }\OtherTok{=} \FunctionTok{count\_sequence}\NormalTok{(}\FunctionTok{simulate\_player}\NormalTok{(}\FloatTok{0.5}\NormalTok{))}
\NormalTok{\}}
\FunctionTok{hist}\NormalTok{(x, }\AttributeTok{col=}\StringTok{"white"}\NormalTok{, }\AttributeTok{main=}\StringTok{"Distribution of longest streaks"}\NormalTok{,}
     \AttributeTok{xlab=}\StringTok{"size of longest streak"}\NormalTok{, }\AttributeTok{ylab=}\StringTok{""}\NormalTok{)}
\FunctionTok{abline}\NormalTok{(}\AttributeTok{v=}\FunctionTok{mean}\NormalTok{(x), }\AttributeTok{lty=}\DecValTok{2}\NormalTok{, }\AttributeTok{col=}\StringTok{"red"}\NormalTok{, }\AttributeTok{lwd=}\DecValTok{2}\NormalTok{)}
\FunctionTok{text}\NormalTok{(}\DecValTok{10}\NormalTok{, }\DecValTok{1000}\NormalTok{, }\FunctionTok{paste}\NormalTok{(}\StringTok{"mean = "}\NormalTok{, }\FunctionTok{round}\NormalTok{(}\FunctionTok{mean}\NormalTok{(x), }\DecValTok{2}\NormalTok{)))}
\FunctionTok{text}\NormalTok{(}\DecValTok{10}\NormalTok{, }\DecValTok{700}\NormalTok{, }\FunctionTok{paste}\NormalTok{(}\StringTok{"min = "}\NormalTok{, }\FunctionTok{min}\NormalTok{(x)))}
\FunctionTok{text}\NormalTok{(}\DecValTok{10}\NormalTok{, }\DecValTok{400}\NormalTok{, }\FunctionTok{paste}\NormalTok{(}\StringTok{"max = "}\NormalTok{, }\FunctionTok{max}\NormalTok{(x)))}
\end{Highlighting}
\end{Shaded}

\includegraphics{ps1_template_files/figure-latex/unnamed-chunk-4-1.pdf}

\hypertarget{c}{%
\subsection{2.C}\label{c}}

\begin{Shaded}
\begin{Highlighting}[]
\NormalTok{pHits }\OtherTok{=} \FunctionTok{seq}\NormalTok{(}\AttributeTok{from=}\FloatTok{0.1}\NormalTok{, }\AttributeTok{to=}\FloatTok{0.9}\NormalTok{, }\AttributeTok{by=}\FloatTok{0.05}\NormalTok{)}
\NormalTok{mean\_pHit }\OtherTok{=} \FunctionTok{c}\NormalTok{()}
\NormalTok{max\_pHit }\OtherTok{=} \FunctionTok{c}\NormalTok{()}
\NormalTok{min\_pHit }\OtherTok{=} \FunctionTok{c}\NormalTok{()}
\NormalTok{x }\OtherTok{=} \FunctionTok{c}\NormalTok{()}
\NormalTok{curr\_pHit }\OtherTok{=} \DecValTok{1}
\ControlFlowTok{for}\NormalTok{ (pHit }\ControlFlowTok{in}\NormalTok{ pHits) \{}
  \ControlFlowTok{for}\NormalTok{ (i }\ControlFlowTok{in}\NormalTok{ (}\DecValTok{1}\SpecialCharTok{:}\DecValTok{10000}\NormalTok{)) \{}
\NormalTok{    x[i] }\OtherTok{=} \FunctionTok{count\_sequence}\NormalTok{(}\FunctionTok{simulate\_player}\NormalTok{(pHit))}
\NormalTok{  \}}
\NormalTok{  mean\_pHit[curr\_pHit] }\OtherTok{=} \FunctionTok{mean}\NormalTok{(x)}
\NormalTok{  max\_pHit[curr\_pHit] }\OtherTok{=} \FunctionTok{max}\NormalTok{(x)}
\NormalTok{  min\_pHit[curr\_pHit] }\OtherTok{=} \FunctionTok{min}\NormalTok{(x)}
\NormalTok{  curr\_pHit }\OtherTok{=}\NormalTok{ curr\_pHit }\SpecialCharTok{+} \DecValTok{1}
\NormalTok{\}}
\FunctionTok{plot}\NormalTok{(pHits, mean\_pHit, }\AttributeTok{col=}\StringTok{"blue"}\NormalTok{, }\AttributeTok{pch=}\DecValTok{16}\NormalTok{, }\AttributeTok{xlim=}\FunctionTok{c}\NormalTok{(}\FloatTok{0.0}\NormalTok{, }\FloatTok{1.0}\NormalTok{), }\AttributeTok{ylim=}\FunctionTok{c}\NormalTok{(}\DecValTok{0}\NormalTok{, }\DecValTok{25}\NormalTok{),}
     \AttributeTok{type=}\StringTok{"o"}\NormalTok{, }\AttributeTok{xlab=}\StringTok{"prob score"}\NormalTok{, }\AttributeTok{ylab=}\StringTok{"streak length"}\NormalTok{)}
\FunctionTok{points}\NormalTok{(pHits, max\_pHit, }\AttributeTok{col=}\StringTok{"green"}\NormalTok{, }\AttributeTok{type=}\StringTok{"l"}\NormalTok{, }\AttributeTok{lty=}\DecValTok{2}\NormalTok{)}
\FunctionTok{points}\NormalTok{(pHits, min\_pHit, }\AttributeTok{col=}\StringTok{"red"}\NormalTok{, }\AttributeTok{type=}\StringTok{"l"}\NormalTok{, }\AttributeTok{lty=}\DecValTok{2}\NormalTok{)}
\FunctionTok{legend}\NormalTok{(}\StringTok{"topleft"}\NormalTok{, }\AttributeTok{legend=}\FunctionTok{c}\NormalTok{(}\StringTok{"mean"}\NormalTok{, }\StringTok{"max"}\NormalTok{, }\StringTok{"min"}\NormalTok{), }\AttributeTok{bty=}\StringTok{\textquotesingle{}n\textquotesingle{}}\NormalTok{, }\AttributeTok{col=}\FunctionTok{c}\NormalTok{(}\StringTok{"blue"}\NormalTok{, }\StringTok{"green"}\NormalTok{, }\StringTok{"red"}\NormalTok{),}
       \AttributeTok{lty=}\FunctionTok{c}\NormalTok{(}\DecValTok{1}\NormalTok{, }\DecValTok{2}\NormalTok{, }\DecValTok{2}\NormalTok{))}
\end{Highlighting}
\end{Shaded}

\includegraphics{ps1_template_files/figure-latex/unnamed-chunk-5-1.pdf}

\end{document}
